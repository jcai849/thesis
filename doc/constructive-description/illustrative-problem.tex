The working statistician is presented with a staggering array of varying
datasets from which to attain insight. The data varies along many
different fronts, including complexity, shape, velocity, and size, among
others. In this chapter, we seek to compose a motivating example with
which to illustrate the development of our answer to the problem of
large data.

As an example dataset, we consider only data which exists beyond the
regular spectrum of size, sitting larger than memory. Concretely, for
the purpose of example, consider the NYC TLC dataset. This dataset
consists of measures relating to all taxi and limousine trips in New
York City from January 2009 to present. There are roughly 2 billion rows
and 25 columns, totalling over 60Gb. Measures include aspects of the
trip such as pickup and dropoff times and locations, passenger count,
tips and tolls, and vendor. The dataset is notoriously messy for such a
large tabular format; locations are censored from midway through the
dataset for privacy reasons, many fields depend entirely on others, etc.
It is these qualities in particular that make it valuable as an example
for analysis; most tools don't make it easy for users to explore such
issues in data, and analysis is hindered - an outcome of this project
should be to make such exploration easy.

With such a dataset, many questions may be asked of it. Geographic
questions, such as where hotspots in dropoff and pickup are, or if there
is any clustering when looked at in tandem. Behavioral questions, such
as tipping and the relation to other variables. The dataset has many
variables that could serve as foreign keys with which to link to other,
smaller datasets, in order to gain more insight, such as local events
taking place on particular dates, etc.

A full analysis is to be performed on the dataset, involving all the
standard components: hypotheses, exploratory data analysis, etc.,
culminating in the use of some advanced, as-yet unimplemented
statistical technique. Given the previous methodologies explored in the
previous literature review chapter, let's consider in detail just what
such a technique should be; notably, it should serve to give issue to
all of them, as a prompt to develop upon for our platform. Ideally, this
would be a research-level algorithm for statistical analysis, having no
pre-existing implementations offered at the user level by any of the
platforms explored. There are some issues with such an approach,
however, in that cutting-edge statistical techniques require a highly
localised understanding and are typically extremely complex, with
implementation being very specific to the technique. This brings the
risk that the illustration will draw more attention to itself than to
the platform it seeks to illustrate. As such, we instead make use of an
approach well-known to statisticians, the Generalised Linear Model,
implemented through Reweighted Least Squares. Such a model is
well-understood, and the implementation sufficiently straightforward for
the sake of clarity, and possessed enough features which enable it to be
analogised to a broad variety of other statistical algorithms. The
intention is to allow this example to stand in for the use-case of a
research algorithm as applied to the dataset, with appropriate analogies
to be drawn throughout. In later chapters, following the construction of
such the platform, it will be put to use in the actual implementation of
advanced statistical algorithms, to prove the concept. Therefore, we
consider as our motivating example the NYC taxi dataset, to be analysed
using a GLM, as the proxy for an unimplemented statistical modelling
algorithm.

With such an example as a background illumination, we will build from
scratch a tool which can operate in a way that no other can, and perform
complex analyses with ease.

Our first task will require consideration of the very
\textbf{representation} of the data. Being such a large dataset, it will
necessarily pose problems to any naive implementation. Furthermore,
assuming the existence of some means of representing the data, we will
need to operate upon it, in order to gain any insight from it - this is
at the \textbf{high level}. At a \textbf{lower level}, we will need to
determine the operations available to us in order to easily and
efficiently implement the complex algorithms with which we use to
perform the analysis. Importantly, it should be just as easy to
implement a new statistical algorithm as it is to use a pre-implemented
one.

For example, consider that we have the data as it's raw form directly
from the TLC website, as a collated csv file on disk. How do we begin?

\begin{listing}
    \begin{minted}{r}
massive_data_crash_r <- read.csv(tlc_data)
\end{minted}
    \caption{Naive read of larger than memory data guaranteeing a crash}
    \label{lst:massive-crash}
\end{listing}

\ref{lst:massive-crash} simply won't work. The object in memory will crash even high-end
PC's, at its 60Gb+ size. There must be a better means of representing
the data than its direct form as standard R objects. Conversely, such a
representation can't possibly stray too far from what the statistician
is familiar with in the language. Assume, however, that it does somehow
exist, and can be treated like a regular object. For example, we may
have it represented as a dataframe named \mintinline{r}{nyc\_taxi.df}, with
columns to be accessed through the \mintinline{r}{\$} operator and the name of
the column. How do we perform basic operations upon it? For instance,
how might we determine the largest tip given?

\begin{listing}
    \begin{minted}{r}
        max(nyc_taxi.df$tips)
\end{minted}
    \caption{Typical determination of maximum in R}
    \label{lst:max-tip}
\end{listing}


\ref{lst:max-tip} is the standard means of interaction, but this operation is highly
dependent on the means of data representation, and will require some
discussion of the scaffolding required to make it work. And based on
this, what are the operations that can be depended upon in order to
create such scaffolding? In partial response to the first statement,
there is some knowledge at least of data representation - that of
smaller subsets of data, which may be read into memory. These subsets
allow for some operations to be performed, each providing a window into
some part of the whole dataset. With all subsets of a dataset operated
upon independently, for some operations this is equivalent to performing
an operation over an entire dataset. These subsets are referred to as
\textbf{chunks}, and will be described in greater detail in the
following chapter. For now, consider the chunks as containers for a
subset of the data, which in itself requires certain operations in order
to interact with. For instance, some operations required are to
unmarshal the data out of the chunk, and perform operations on data
within the chunk - reading and writing, as minimum necessary features.
The \textbf{relation between chunks} within a \textbf{group of chunks}
at a higher level is also worth bearing some thought. For instance, when
a set of chunks are treated as a cohesive object, and the underlying
data of the object is sought, how best to retain the shape of the
dataset cohesively, without leaving any distortion from the chunks? For
example, given that at a high level, chunks are to be forgotten, how
best to avoid artifacts of subset formation? Consider \ref{lst:chunk-object}
where \mintinline{r}{d1} refers to a group of three chunks treated as one
object:

\begin{listing}
    \begin{minted}{r}
print(d1)

# 3 chunks
# chunk 1: int 304
# chunk 2: int 12348
# chunk 3: int -5899
\end{minted}
    \caption{Group of chunks in a collection as a single object}
    \label{lst:chunk-object}
\end{listing}

The \mintinline{r}{d} stands for ``\textbf{distributed}''; chunks must not
occupy memory all together, otherwise memory will run out -- therefore,
they must be distributed across a set of computers, or accessed
sequentially within one computer. Each chunk holds only one single
integer for the sake of simplicity of example. When the integers are
unmarshalled out of each chunk, or ``\textbf{emerged}'', it may be
expected that a cohesive integer vector of length three should result,
here given as variable \mintinline{r}{l1} in \ref{lst:emerged-chunks}. In this case, \mintinline{r}{l} stands for
``\textbf{local}''; that is, local to the operating environment, as any
regular object, without the indirection created by chunks.

\begin{listing}
    \begin{minted}{r}
l1 <- emerge(d1)
print(l1)

# int [1:3] 304 12348 -5899
\end{minted}
    \caption{Chunks emerged to a cohesive small object}
    \label{lst:emerged-chunks}
\end{listing}

Were the chunks to be treated as mere containers for a single underlying
object, this is the desired behaviour. Failing this, without a means of
defined relation between a group of chunks, without any manner of
combination, the following behaviour may result instead, where a group
of emerged chunks retains the structure imposed by chunking, as a list
or similar, as in \ref{lst:emerged-chunks-structured}:

\begin{listing}
    \begin{minted}{r}
l1 <- emerge(d1)
print(l1)

# List of 3
#  $ subset1: int 304
#  $ subset2: int 12348
#  $ subset3: int -5899
\end{minted}
    \caption{Chunks emerged to a small object which retains the chunked structure}
    \label{lst:emerged-chunks-structured}
\end{listing}

These requirements can be summarised as the considerations concerning an
object system, and are therefore described in greater detail in
\ref{sec:object-system}.