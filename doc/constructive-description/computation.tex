With the establishment of an object system, attention may then be turned to the actualities of computation.
Some questions that have been brewing may be investigated: What form does a computation possess?
With the intention of the system to allow for extension at every level, what is a necessary minimum basis with which to engage at a level lower than the aforementioned ``chunk'' layer, where the very operations that are performed on chunks are concerned?
What can a developer expect to take place in the system when an operation is performed on some object?

Such questions depend on further elucidation of the system, including determination of the specific means of distribution of the chunks.
While this dependency brings much light to the subject, this will be delayed in order to constrain the discussion to only those features of computation that are invariant to distribution specifics, with these details being discussed later.

We may continue with the \code{max} example as raised in the preceding section, with greater focus on the computational aspects.
Starting again, we have some chunk referenced through \code{d1} in \cref{lst:chunk-d1}:

\src[Rout]{chunk-d1}{Single chunk for manipulation}

We want to attain the maximum of this chunk.
As given before, we may run the \code{max} function over it, returning another chunk reference to a chunk representing the \code{max} of \code{d1}, and emerge the result to attain the max as a local, standard, object in \cref{lst:emerge-max}:

\src[Rout]{emerge-max}{Emerging the maximum of a chunk}

Let's look at this from another angle, tracing the very route of the computation that creates such a result.
The anatomy of a computational function may guide discussion, with the parts being arguments, body, and returned value.
Applied to chunk operations:

\begin{itemize} \item Operations on chunks must have some means of access to all of the relevant chunk arguments to operate on together; \item the operation must have a stored description of some kind; \item and the result of the operation must be stored somewhere.
\end{itemize}

With respect to the \code{max} operation above, the first point means that the chunk pointed to by \code{d1} must be located in the same memory space as what the \code{max} operation has direct access to.
This means that the computation must take place remotely.
The remote memory space which has operations performed is known as a \textbf{worker}, which may possibly be a distinct computing node in a network.
This contrasts with the local node which holds the chunk references and is interacted with directly, which is known as the \textbf{client}.
Here a specific choice is made as to where the computation takes place; the \textbf{computation travels to the chunk}, rather than the chunk travelling to the computation.
A corollary of this fact is what allows for computations to take place over multiple chunks which can't coexist in memory: that each computation takes place within a separate memory space.
Furthermore, the computation, as will be described, is typically several orders of magnitude smaller in size than the chunk, and is therefore more efficient to transport.

To investigate this, let us first introduce the notion of a function applicator that sends a function to be run over chunks, through their chunk references.
We label it \code{chunk-call}, taking some function to be applied as its first argument, with the remaining arguments chunk references to have the function remotely applied to, and returns a single chunk reference to the result.
For example, we have a potential implementation of remote \code{max} at a low level using \code{chunk-call} in \cref{lst:chunk-call}.

\src[Rout]{chunk-call}{Directly calling a function on a chunk}

The use of \code{chunk-call} on a single chunk may be effectively useless if the remote memory space the chunk occupies is equivalent to the local space - in this case it would be simpler to run the computation directly, locally, and avoid any overhead associated with the indirection provided by chunks.
When multiple chunks are to be considered for a computation, the situation changes; With the size of all chunks together exceeding memory limits, the worker performing the operation must only possess some of the chunks in memory at any one time.
While seemingly paradoxical, if the set of permitted operations is sufficiently constrained to only allow for some of the chunk arguments to be operated on together simultaneously, then memory limitations may be respected.
An example may be a parallel arithmetic operation on two vectors of equal length; At any point in the specific arithmetic operation, all that is required are the two elements existing at equal indices of the vectors.
Some number of elements that remains within memory bounds may then form a chunk, with each corresponding set of chunks may be operated upon in memory, until an entire set of chunks, larger than memory in aggregate, may be processed.
Take for example, a slightly modified version of the previous example, where rather than a single maximum, a running parallel maximum along several vectors of equal length is to be computed.
Each vector is larger than memory, and as such, is broken into several chunks.
These chunks can be compared across vectors in memory, without the entirety of the vectors being loaded into memory together.

Concretely, consider vectors \code{v1}, \code{v2}, and \code{v3}, each of which is composed of one trillion integers, and the running parallel maximum is to be found amongst them.
Each vector is broken into ten thousand equally-sized chunks, with the chunks labelled according to vector \code{i} and chunk \code{j}, as \code{dij}, so that the third chunk of the first vector is referred to as \code{d13}.
Then each \(j\)th chunk of the vectors can be compared across with \code{pmax}, using \code{chunk-call}, with the \cref{lst:call-along-chunks} comparing the first chunk of each of the vectors:

\src[Rout]{call-along-chunks}{A call along multiple chunks}

Were each vector to be taken to the user level, as described in the previous section, they would be treated as distributed objects.
The first vector \code{d1} may look something like the \cref{lst:distobj}:

\src[Rout]{distobj}{An example distributed object}

Here we may introduce a higher-level version of \code{chunk-call}, which similarly applies a function given as argument to the remainder of arguments, though the remainder of arguments in this case are to be distributed objects.
With the example given above, we may label this function \code{distobj-call}, and run it with \code{pmax} over the distributed object vectors \code{d1}, \code{d2}, and \code{d3} like \cref{lst:distobj-call}:

\src[Rout]{distobj-call}{Calling over a distributed object}

With this example, the means of access to the multiple arguments also raises questions; were the chunks distributed across a network, the worker performing the computation requires some \textbf{means of attaining the relevant chunk arguments}, and if they were to be operated upon in a stream, a consistent method of pulling chunks into memory is required.
Here the water muddies with respect to moving computations to data, versus moving data to computations.
But with such a methodology, it is clear that the computation always moves to some data, with the remainder of data movement occurring only as required.
Such further data movement can be directed through a variety of strategies, including optimising to minimise size of movement, optimising to minimise load on the end worker, among others - none are to be specified at this stage.

One situation that would have to be considered is the mixing of local and remote arguments.
For instance, returning to our earlier, simpler example in \cref{lst:earlier-example}:

\src[Rout]{earlier-example}{Finding the maximum of a distributed object}

Nothing on the face of it is wrong with such a statement, as indeed the semantics are exceedingly clear.
A chunk is to be operated on, and the operation is to take the local argument \code{na-rm} A threshold has been crossed however, in now allowing for the \textbf{mixing of local and remote arguments}.
The worker requires all relevant arguments to the operation to exist in its own memory.
Therefore, \code{na-rm} must be sent to the worker as part of the operation.
For the sake of consistency, it may be that \code{na-rm} is sent over to the worker to be treated as a chunk, including the generation of a local chunk reference -- but this is to be decided by the implementation.
There at least seems a clear answer to such a situation, though it is important to consider the very existence of such a situation.

The situation of argument attainment complexifies enormously when \textbf{arguments of differing length} are supplied.
The above example had a very obvious interpretation, where \code{na-rm} is sent as-is to wherever the chunk referred to by \code{d1} is located.
Even in a situation like the \code{distobj-call}, run with a \code{max} over a single distributed object, it is obvious that \code{na-rm} should be sent to all workers operating on each chunk of the distributed object.
That is, obvious to the user, but not necessarily so obvious to the computer.
It is obvious to the user because the user knows of it as a scalar.
In other situations, the scalar may be treated as a vector to be recycled to the same length of the greater object.
For instance, the \code{pmax} between a scalar and a vector.
Recycling is not always intended, and when comparing distributed objects of differing length, different functions may require different strategies.

\cref{fig:scalar-index} shows precisely this example.
Here, \code{x} is a chunk containing a single integer, \code{1}, and \code{y} is a distributed object composed of three chunks, each holding a section from \code{1up}.
In all chunk computations with \code{max-x-y}, \code{x} is to be recycled to engage as an argument with all chunk operations involving \code{y}.
Conversely, if \code{y} is to be indexed by \code{x}, it should not be recycled, and each chunk belonging to \code{y} should receive different transformations of \code{x}.
Specifically, the first chunk of \code{y} should receive \code{x} as-is, because with \code{x} as \code{1}, it selects the first element.
The first element of the distributed object is the first element of the first chunk, but nothing should be selected from the other chunks.
Were a \code{1} to be indexed from each of the other chunks, they would index relative to themselves, and unintended results would ensue.

\widefig{scalar-index}{Figure showing a chunk that is treated as-is and recycled in the situation where it is considered a scalar, but converted to something else entirely when used as a numerical index}

The earlier section implies that length is always known of a distributed object, which hasn't necessarily been established.
Were the object under consideration the result of a long chain of operations which may alter the length, and the chunks all remote, there is not any local knowledge of length without a direct query to the relevant workers and a summation of all of the lengths.
To add to the difficulty, simple vectors with a single ``length'' value are not the only underlying datatypes that are allowed for, as specified in the Object System section; the data may be of any dimension, and indeed any shape, making the idea that recycling can take place appear increasingly impossible.
Indexing of a distributed object is a similar problem, for similar reasons - there is no theoretical solution to a completely open problem as this, so experience through prototyping will have to be relied upon to provide some reasonable heuristics and defaults.

With the arguments in place for the computation, the computation can be run on the worker.
What computation?
Specifically, how does the worker know of the computation, and how is the computation run?
The implementation details of how the worker receives the computation may vary, however an essential aspect is the serialisation and sending of the procedure as supplied to one of the applicator functions on the local end to the chunk reference.
The important step, made opaque by the applicator functions, is to couple the reference to the chunk arguments with a computation, as a set of instructions, and send that to the worker.
This step implies two additional corollaries:

\begin{itemize} \item References to chunks must be understood by both the local memory as well as the remote worker memory.
	      This means that there should be some universal means of reference to specific chunks within the system; either through tagging or other ID measures.
	      Were this not the case, the worker would not understand what chunk is being referred to as part of the computation instructions as sent from the client.
	\item
	      Computations are sent to the worker, which would ideally have most of
	      the relevant data available.
\end{itemize}

After receiving the packaged computation and chunk references, the worker would first have to arrange the chunks in its memory as described above, and with all this in place, deserialise the provided computation and run it over the chunks in a regular fashion, with all relevant pieces being local.
Despite the running of a computation on remote chunks being core to the running of the system as a whole, this aspect appears to be the most straightforward of any.
Experience will put such a statement to the test.
One possible difficulty lies in the fact that arguments are not the only relative data to the running of any function.
Functions may reference variables beyond just the arguments, such as global variables.
Were this to be allowed in the system, a means of identifying such variable use in the body of a function is required, as they would have to be sent alongside the arguments.
This may add massive amounts of complexity to each remote computation call however, as such static analysis, particularly of dynamic languages, is notoriously difficult and unreliable.

Following the running of the computation, the ``body'' as given in the earlier guide to this discussion, a returned value results.
There are two aspects worth considering with respect to such a value:

\begin{itemize} \item How it is stored as a chunk on the worker \item How it is referenced on the originator of the computation (the client) \end{itemize}

Both aspects depend on the notion that chunks must persist, which is a point worth focussing on directly: First, consider the alternative, where chunks are not persistent.
In such a form, the chunk resulting from any remote operation would have to be sent immediately to the client in order for anything useful to eventuate from its creation, assuming such an operation was not run purely for its side effects.
Such is the course taken by \pkg{SNOW}, which has such limitations already addressed.
Notably - the enormous amount of movement back and forth of data, and unintuitive programming techniques to emulate persistence; neither of which are suited to very big data.
As such, let it be assumed that chunks persist.
Returning to the two aspects just described, the question is: How?

Consider first how the result is stored on the worker.
This would be specific to the form of distribution of chunks.
Were chunks to be distributed over time across the same machine, sinking to disk is the only possible option, with the maintenance of some further reference to the location on disk.
With chunks distributed across a cluster of nodes, the same option is available, as well as the potential to maintain the chunk in-memory, assuming each chunk is sufficiently small enough to leave space for other chunks, which could be enforced.

Reference of the result on the client requires some agreement at the time of sending the computation request as to how the result is to be referenced.
Were the request to be sent without any contract between client and worker as to how to reference the result, any further communication would also lack this information, as the result would be indiscernible from any other computation result.
Therefore, some initial contract as to how to refer to the resulting chunk is required between client and worker, with a variety of strategies discussed in the appendix.
Irrespective of strategy, the persistence of chunks dictates that the result should be a chunk, with the returned value on the client a chunk reference.
With this consistency, it can be seen inductively that indefinitely long chains of remote operations can take place without need for continual data movement between client and workers.
For example, \cref{lst:type-annotated} gives an annotation of the type of each object as it may exist relative to the client node:

\src[Rout]{type-annotated}{Relative locations of each object}

Here, the variable \code{a} is of particular interest, as it requires the creation of many temporary chunks as each addition computation is engaged in.
The treatment of such temporary chunks is worth consideration, as without appropriate collection, they will gather and leak memory.

With client and workers engaging in multiple computations, it is now worth asking how they are to be arranged over time.
The following section seeks to explore precisely this question.
