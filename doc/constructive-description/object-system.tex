In order to perform calculations on larger-than-memory data, we need
some means of \textbf{representing} the data, in order for it to be
tangible and useful. Let us start at the beginning, where we accept that
some region of memory in the computer must hold a form of the data. We
label such a region an \textbf{object}, and maintain that it is general
enough that it could very well be a function, a variable, an instance of
a class, or a reference to another object, among a multitude of other
possibilities. Given that the whole data is larger than memory, it can't
exist as a regular object taking up some contiguous region of memory,
and must necessarily take some other form, which at the moment can be
left for later description. Irrespectively, the data exists as an
abstract object, albeit one that is somewhat irregular.

With respect to our motivating example, the NYC taxi data must
necessarily exist in memory somewhere.

All objects, regular or irregular, are facilitated in their access and
manipulation by way of an \textbf{object system}, which is an organised
manner of interaction with objects. We choose the description of an
object system as the basis for the theoretical development of our
project as it is that aspect which is most proximal to the constraint
imposed by the scale of data dealt with, in being that aspect which
defines how data is dealt with. We may begin this description by asking
what may be intended of an object system within the scope of this
project. In order to answer this, we must first understand precisely who
the audience is for this project. Different audiences necessarily seek
different interactions with objects, through their differing emphases
and required end-results. It may be that multiple audiences require
multiple object systems, or that one object system may be sufficiently
general to please all users.

With the consideration that our audience is to be working statisticians,
as per the initial scope, there still remains some variation within this
greater audience that can be described as separate audiences. A notable
split is that of depth of use of the program provided by the project.
Depth of use separates developers, who will take the physical form of
the data into consideration, from users, who simply want to use the
data. Alternative splits exist, and a pattern emerges among them of a
distinction between power-users and end-users. Without holding this
split as the defining audience differentiator, its existence proves that
more than a singular monolithic audience needs consideration.

Therefore, the object system must engage multiple audiences. This
endeavour may take several different forms, through the existence of
multiple object systems, or one singular object system. Such a component
will necessarily be driven by the experience of use, but it is
worthwhile to explore these options in greater detail prior to
implementation. Considering a single object system first, there is the
difficulty of appeasing multiple audiences simultaneously. This object
system may favour one set of audiences over another, which in turn
decreases the value of this project for those audiences unfavoured by
the system. Using multiple object systems, with each geared to a
particular audience, may overcome such a difficulty, though raises
another question, in how the systems relate to each other. As an example
solution, with power-users and end-users as illustrative audiences,
consider the standard approach taken within software engineering, of
\textbf{layering}. In this case, the power-user sits at a layer below
that of the end-user, in the sense that the power-user is closer to the
data with less abstraction. The abstraction is then what changes the
object system of the power-user into one interfaced with by the
end-user. The assumption in this case is that the end-user object system
builds upon that of the power-user. This aligns with standard software
engineering practice to build more user-friendly tools out of more
powerful ones. This layering effect also provides a separation of
concerns, where separate concerns (logical components) are entirely
encapsulated from each other. The separation enables the program to be
developed independently at each layer, aiding maintenance as well as
design.

Such separation of concerns and layering in the object system may take
it's embodiment in the representative objects offered by each layer. The
highest layer is that in which the user has no concern for the
representation of the object on the disk. Here, the object is no more
than a vessel for whatever data the user seeks to hold, with no thought
toward the implementation details. Such an object would define the
higher layer of the object system, and would necessarily be supported by
lower layers. Compare this with the lowest layer, in which the
implementation of this data container could be manipulated directly.
This requires a consideration of the nature of the objects, now that
implementation details are important. The central constraint is that the
data is larger than computer memory. This, coupled with the fact that
they must be worked on in-memory, leads to an impossible design if
combined naively. The only possible means is to split the data into
smaller pieces that do fit in memory. Such a structure, common among big
data systems, is known as a shard, or a \textbf{chunk}. This is the only
possible result to such a limitation, though it may have many
manifestations; we leave such details of implementation to follow
experimentation, keeping the description as high-level as possible for
now.

In order to understand chunks, the notion of \textbf{references} must be
first understood: A reference is some object that acts as a means of
connection to some other object, known as a \textbf{referent}. The
reference is a value in itself, and is likewise referred to by some
symbol, as any other object. The reference serves as a means of
indirection, and takes many forms; for instance, the hyperlink, or the
pointer. References may be ``dereferenced'' in order to access the other
object that it leads to as a value - clicking the hyperlink, or
\ccode{star}ing the pointer.

On a system where referents do not necessarily exist in the same memory
space as the reference, we may say that the reference is \textbf{local},
and the referent \textbf{remote}. Upon being dereferenced, the referent
is pulled into the same memory space as the reference and therefore also
becomes local. Why would references be used, when direct access to the
underlying object may be thought more straightforward? One reason is
specific to the local-remote distinction; as the only possible means of
interacting with remote objects.

For instance, were subsets of a too-large-for-memory dataset to be
interacted with, they must be distributed, over space or time. Over
space, each subset may sit on a separate node in a computing cluster,
within it's own memory. Over time, each subset may be pulled into the
memory of a single node and operated on sequentially, without all
subsets existing in the same memory space simultaneously. Each subset is
therefore remote to whatever may be controlling their total operations
from some central position, and references provide a means of access to
each subset. Chunks in all systems are likewise dependent on references,
for their capacity to stand in as proxies to the underlying data subset.

The chunk thus serves as the lowest level object manipulable by the
user, and defines the lower level of the layered object system. Based on
the lower layer serving as a basis for the higher layer, chunks would be
used to create the abstract object interacted with by the end-user
unconcerned with implementation details; the end-user not interested in
the chunks making the object. With the high-level object composed of
chunks, it therefore forms some variation of a \textbf{distributed
    object}. Whether the chunks are necessarily physically dispersed, or
accessed from the same location at differing timepoints is a question to
be settled later - the overall object is irrespectively distributed over
space or time. The power-user would therefore have access to chunks, and
the end-user access to the objects composed of chunks, a generalised
distributed object, whose manner of distribution may remain undefined
for now. Within this range, there may be intermediate objects. For
instance, there may be some object which serves as a container for
chunks, though still retains access to implementation-specifics of
chunks, without hiding such information as the abstract distributed
object does. Such an object may be defined by widely differing container
shapes, such as a vector, matrix, array, or some other nonlinear form.

Coming back to the example analysis introduced earlier, we may consider
that the NYC Taxi data, being a heterogeneous table, would find its
optimal existence as an abstract dataframe. Implementation-wise, the
underlying chunks would have to split along some dimensions of the
dataframe. Given that each row is independent, splitting along rows is a
natural point of separation. Were the size of each chunk small enough to
fit in memory, this would serve as a sufficient description for
chunking. Assuming that the dataset is stored somewhere on disk, at the
low level it would have to be read in as chunks. This may take the form of
\cref{lst:read-chunks}:

\src{read-chunks}{An example syntax for reading in chunks}

All of the chunks can't simultaneously exist in memory on one single
computer, for the same reason that the entire dataset can't. Thus this
collection of chunks will be a set of pointers to pieces of the dataset
across time or space. This low-level form may be abstracted over in a
\ccode{read-csv}, or \ccode{dbconnect}, serving as just one particular
method which allows for chunks.

The key result of such objects, beyond the multiple audiences allowed
for, is that potentially \textbf{arbitrary data structures} may be held
in chunks, or distributed objects. This underlying data is therefore
able to be specified at the chunk level, and were the system to allow
for arbitrary data, special means would be required to interact with it.
Specifically, the underlying data will at some points be interacted with
directly, and for the system to maintain sufficient generality that the
data may take any form, the interaction points would have to be
well-specified and extensible to allow correct behaviour at interaction.
This would have to be a specifically considered and described component
of the object system. Experience will determine how this may best be
afforded, whether through polymorphism through an object-oriented
system, untyped procedures, or various other means.

Some basic necessities of interaction are reads and writes. Reads are
means of access to the data underlying chunks. While it is not possible
to access the full dataset directly, each chunk may be summarised to a
degree sufficiently small enough to read in and use as a regular object,
through some massive dimensionality reducing operation, such as a
\ccode{max} or \ccode{sum}. For example, when looking for a
\ccode{max}, a chunk consisting of a billion integers is transformed
into a chunk of a single integer. This single integer may be pulled out
of the chunk and managed as a regular integer, through some read
operation, which we name an \textbf{emerge}. The operation is equivalent
to accessing the value of some data held in a container such as a list,
though in this case the important distinction is the movement of the
value from remote to local memory space. Take the example code of \cref{lst:chunk-container}
as illustration:

\src[Rout]{chunk-container}{An example of chunk as container}

Here, we have the symbol \ccode{d1} specifying a reference to some
chunk, which is a subset of some unspecified greater whole. The chunk is
an integer vector of length one billion and exists remotely, while the
reference to the chunk is local and immediately accessible through the
symbol \ccode{d1}. To interact with the integer vector directly, it may
be emerged, which pulls the vector into local memory space as a
standard, regular, integer vector. In the \cref{lst:assign-emerge}, we assign
the emerged vector to \ccode{l1}; that is, ``local''\ccode{1}.

\src[Rout]{assign-emerge}{Reading chunk value through an emerge}

The write operation is the more general complement of actually
transforming a chunk in some manner. It will be discussed in greater
detail in the following chapter, though an example of how some
interaction may be stated syntactically is given in the following
listing; \cref{lst:write-chunk}, the act of maximisation is shown at a high level, with
implicit write operations, and an explicit read operation using
\ccode{emerge}.

\src[Rout]{write-chunk}{Writing new chunk through chunk-capable function}

Note that if run in a functional manner, the object \ccode{d2} is
entirely distinct and separate from \ccode{d1}. The mechanism of such a
transformation is described further in the following chapter.
Importantly, \ccode{d2} is also a chunk reference, with no emerges
having taken place. The chunk referred to may exist in a separate memory
space to the reference, just like \ccode{d1}. No specification is given
as to whether it necessarily exists in the same memory space as
\ccode{d1} however, only that both are remote to the reference.

The above is interaction purely at the pure chunk level. When
considering collections of chunks, or abstracting over them as the user
level would require, certain differences arise. For example, when
seeking the maximum of a set of chunks, due to the transitive nature of
such a function, the maximum of the chunks is the maximum of the
maximums of each chunk. But this is only so if the underlying data
structure possesses meaning for such a function. There must be some
means of mapping from a set of chunks to the entire distributed object
considered as a whole. One example was hinted at before, where chunk
references are held together in an array, with a defined underlying
datastructure making up the chunks as pointed to. Operations over a
group of chunks may take a similar form to operations over a collection
of other objects, which are commonly referred to as \ccode{apply}
functions, both in \R and in the wider computer science world. A means of
operating over a group of chunks may be given in a \ccode{chunk-apply}
function, acting as a higher-order function that orders the running of
whatever function argument is given, over the set of chunk arguments.
Take the earlier example of maximisation in listing \cref{lst:chunk-collection}, with new object \ccode{d3}
standing for a collection of chunks, being three chunk references, each
pointing to it's own chunk of length one billion; The distributed object
is therefore a three billion length integer.

\src[Rout]{chunk-collection}{Collecting chunks into one object}

In \cref{lst:chunk-apply}, over each chunk, send the \ccode{max} function, to be performed
remotely - this is distinct from the earlier example of pulling in the
data locally first and then operating, with the order here being
reversed. The maximisation takes place in the separate memory spaces
where each chunk resides, possibly on other nodes. The result of the
operation, \ccode{dmaxes}, consists of the remote results of the
operations, itself being a distributed object of three chunks. Each
chunk in \ccode{dmaxes} is the maximum value of each of the respective
\ccode{d3} chunks, as one scalar integer each, equivalently a
distributed object of three integers.

\src[Rout]{chunk-apply}{Application of function over chunks}

Each chunk is now sufficiently small to bring into local memory, and may
be emerged as a single cohesive vector, stored in the local variable
\ccode{lmaxes}, as in \cref{lst:chunk-summarised}

\src[Rout]{chunk-summarised}{Emergence of summarised chunks}

With \ccode{max} being a transitive operation, the max of
\ccode{lmaxes} may be taken in turn, with the operation being performed
as normal, locally, in \cref{lst:total-max}. This resulting value, assigned as \ccode{lmax}, is
equivalently the maximum of the entire distributed object \ccode{d3}.

\src[Rout]{total-max}{The maximum of chunk maxima}

Thus, an operation was performed over a distributed object without the
entire object existing in local memory at any one point in time. It is
clear to see that such behaviour is generalisable to all other
transitive functions that result in sufficiently small summaries that
may be read as a set out of chunks. The \ccode{chunk-apply} function
would sit as a valuable tool for the power-user layer, while
\ccode{max} may have some method defined for distributed objects that
uses \ccode{chunk-apply} in its implementation, and sits at the user
layer.

Let's pursue this further though, with respect to non-singular
summarisations, as this raises further questions. It is easy to conceive
of what form the data from a single chunk should be emerged as. It is
entirely another question as to what the data from multiple chunks,
conceived of as a singular object, should be emerged as. For instance,
consider some function that returns a sample with replacement of a set
of chunks holding numeric data. At the high level, a user for whom
chunks are out of the realm of concern, may simply wish for \cref{lst:high-level} to take place:

\src[Rout]{high-level}{High level functional interaction with chunks}

Here, the results of the operation are automatically emerged from
chunks, and the output of the chunks is combined automatically.
Automatic unmarshalling is straightforward and can be handled in a
wrapper. Automatic combination is less so, as different underlying data
types have different means of combination, and indeed there may be
different intended means of combination for different operations on the
same data type. The means by which such direction can be encoded may
take many forms: It may be polymorphic to be dispatched according to
datatype upon emergence; it may be held as a stored procedure with the
data, or with the chunk; it may be specified manually each time. As with
other facets of the object system, it will have to be tested
emperically, with the contribution made here to recognise its
significance.

For the sake of demonstration, let's continue with the example of
sampling at a lower level, as it will serve to highlight in a more
complex light the importance of the chunk structure and the system
layering, hopefully cementing the concepts explored in this section.
Consider in \cref{lst:varying-sizes} a different distributed object, with chunks of different sizes,
\mintinline{r}{d4} -- consider this as sharing the same first chunk as
\mintinline{r}{d3}, with length \(1 \times 10^9\), but the second and third
chunks split from the greater whole differently, at lengths
\(1 \times 10^7\) and \(9.9 \times 10^8\) respectively.

\src[Rout]{varying-sizes}{A distributed object composed of chunks of varying sizes}

To sample \(n\) elements from this distributed object, each chunk will
have to be sampled from. However, in this case with the lengths of the
chunks differing, the weighting applied to each chunk must vary. Because
the operations on each chunk occur independently and without the
possibility of combining them all for comparison, the remote chunk
samples must occur in isolation. One manner in which such a problem may
be approached is to sample \(n\) times from some probability
distribution over the chunks, then use the results to sample from the
chunks in turn. The chunk probability distribution must be discrete,
with support \(i\) being the integers corresponding to an enumeration of
chunk numbers, and probability of selection for each element in a chunk,
\(p_i\), is proportional to the length \(x_i\) of each chunk. Given the
assumed fungibility of elements in the distributed object, this is
equivalent to a probability-proportional-to-size sampling process, where
the specific probability of selection in each draw is given by \cref{eq:prob-prop}:

\eq{prob-prop}

For this particular example, we have the probability density
function \cref{eq:prob-dens}


\eq{prob-dens}


Which, for \(n=3\) draws, may yield the following series:

\[
    (1, 3, 1)
\]

This implies the first element of the output must be a a sample from the
first chunk, the second element a sample from the third chunk, and the
third element another sample from the first chunk. With the counts of
the series above given as the variable \ccode{chunk-count}, we have
the contingency table \cref{lst:example-sample}, with counts labelled by their
corresponding chunk:

\src[Rout]{example-sample}{Example contingency table of samples}

Treating the table as a vector, we may use \ccode{chunk-apply}, to run
a sample over each chunk, using \ccode{chunk-count} as the \(n\) for
each chunk, as in \cref{lst:sample-apply}. Therefore, the first chunk will run a sample with \(n=2\),
the second chunk will run a sample with \(n=0\), and the third will run
a sample with \(n=1\).

\src[Rout]{sample-apply}{Chunkwise sample application}

This may then be emerged as a local object as in \cref{lst:emerge-sample}:

\src[Rout]{emerge-sample}{Locally emerged sample}

And assuming the existence of some function to rearrange into the order
as given in the original series, perhaps named \ccode{rearrange}, with
the series given as the variable \ccode{series}, we are left with a
random sample with replacement from a distributed object, and none of
the sampling of each chunk necessarily coinciding within the same memory
space as any other chunk, in \cref{lst:random-sample}:

\src[Rout]{random-sample}{The final random sample}

Some considerations that were brushed over relate to what it actually
means to run a computation remotely. How can local arguments be mixed
with distributed, and what does a distributed argument actually mean?
These questions and considerations will be considered in further detail
in the following \cref{sec:computation}, on computation.
