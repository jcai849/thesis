Informed by the interfaces, implementations, and use-cases of large scale statistical systems given by \cref{ch:lit-review}, and the problem of expressing statistical algorithms over larger-than-memory datasets as described in \cref{ch:introduction}, this chapter is intended to provide a constructive description of what form a solution may take.
It is constructive in the sense that it starts \textit{ex nihilo}, adding necessary constraints based on the nature of the problem and the demonstrated negative results that were captured in \cref{ch:lit-review}.
Solutions to aspects are included based on positive results, but the intention with a constructive approach is that local optima may be broken free from, and a broader range of solutions than those existing are considered.
The reason for this is the conceit that there are unexplored solutions that remain to be attempted; the results of this conceit, at least from this project, are given in the later \cref{ch:initial-experiments,ch:ui,ch:implementation}.

The constructive description is initiated through a motivating example, given in \cref{sec:illustrative-problem}.
The intention behind having an example to begin with is that it provides something tangible to consider when engaging in more abstract theorising through the remainder of the chapter.
The system itself is considered in four parts: the object system, computation, concurrency, and reference.
The reasoning for such partitioning is as follows.

A program written in \R{} to implement a statistical solution for a problem involving large data is, ultimately, a computational process.
The two core components of a computational process, split in varying measures, are data and computation\cite{abelson1996structure}.
This notion is carried analogously through the title of \textcite{wirth1985algorithms}, \textit{Algorithms + Data Structures = Programs}, and serves as a common structure for many undergraduate first year computer science curricula.
Data in this case is best covered by consideration of the object system, in \cref{sec:object-system}.
Likewise, computation is considered directly in \cref{sec:computation}.
While this chapter doesn't start with the constraint of distributing data, the essential question of managing multiple things simultaneously -- whether through distribution or parallelisation -- necessitates a description of how concurrency is managed.
This is expanded upon in \cref{sec:concurrency}.
Finally, if distributed systems are the result of this constructive effort, a discusssion of distributed objects are unavoidable.
The use of such data structures entails decisions to be made on the central device of distribution of objects, through reference.
This is given in \cref{sec:reference}.

From the accumulation of decisions that are implied through the constructive description, experimentation may take place in the following \cref{ch:initial-experiments}.
