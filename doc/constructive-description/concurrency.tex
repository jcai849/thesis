After laying out the distributed objects and the operations that they
may be engaged in, we reach the limits of the construction of the
system. The mechanical aspects of the system now developed, we can focus
on the more complex and precise notions of the arrangement of these
components; Let us expand the construction limits now by exploring and
describing the component of time as it relates to the system -
specifically, \textbf{concurrency} within the system. Here, we treat
concurrency in the manner of Pike, where it is used to refer to the
composition of independently executing processes - a very general area
that remains sufficiently directed to fill out the complementary aspect
to the mechanism described so far. We may begin by considering the
system as a whole, and determining the independently executing processes
within it and how they may be composed - this has major ramifications
for the entire architecture. It is clear to see that examination of this
in particular will serve to breathe some life into the form of the
system as we construct it. Each of the grand operations of the system
are themselves composed of smaller operations, and we will likely find
some possible descriptions of concurrency within them, at a lower scale;
notably, chunk operations with respect to both client and worker.

Starting from the top, we may ask what the central operations are that
are to be considered in the overarching composition. With the system
defined thus far through the operations on chunks, it seems wise to
begin there. Given the different points of reference in the system, we
have from the client's perspective the operations of requesting a chunk
operation, and emerging a chunk. A worker sees that of receiving the
order for chunk operations, gathering the chunks, running the operation,
and storing the result. Forcing an even higher-level perspective, we can
group the operations to those chunk operations pertaining to client, and
those to worker. And the question again arises: How to compose these two
independent processes? Stated in the same manner as an section, are
chunks to be distributed over space, or over time? Again, distribution
over space has chunks occupying separate memory spaces simultaneously;
for example they may sit on separate nodes in a computing cluster.
Distribution over time sees chunks occupying the same memory space but
at different points in time, sequentially. There is something of a
gradient between the two options as well, where spatially distributed
chunks may have separate temporal existences as well; for example, a
cluster of nodes that each sinks chunks to disk, pulling them into
memory only when required.

To compare these options, we seek to generalise the problem, and compare
the generalised forms. We claim that a correspondence exists in the
relation of temporal distribution to spatial distribution, to the
relation of concurrency to \textbf{parallelism}. In the manner of Pike
again, we stress the difference between concurrency and parallelism. The
definition of concurrency is given above, and that of parallelism
differs subtly: parallelism is the composition of independent
\emph{simultaneously} executing processes. The difference is subtle but
important, and serves to delineate between \emph{handling} multiple
things at once (concurrency), versus \emph{doing} multiple things at
once. In this case, there is a correspondence between concurrency and
chunk distribution over time, and between parallelism and chunk
distribution over space. Chunks distributed over space all exist in
memory simultaneously - in parallel, whereas chunks distributed over
time don't exist in memory simultaneously, but are necessarily handled
simultaneously.

In comparing the virtues of parallelism with concurrency, the question
effectively boils down to efficiency: If two procedures of equal
processing time requirements are sufficiently independent, enabling
total isolation from each other, they may be run simultaneously and take
half the execution time of running them in sequence. The only additional
time is the overhead of coordination of multiple processes, which is
potentially not all that different from the overhead of organising
processes concurrently. The efficiency gains of parallelism are
well-described in the literature, and clear upon any length of
consideration. The remaining question is that of possibility - many
things likely cannot be parallelised. Any algorithm where each operation
is dependent on the outcome of the previous operation serves as examples
of those where parallelisation is unclear, though it remains an open
question of whether the class of problems that can be solved efficiently
in parallel, \(NC\), is equal to \(P\), the inherently sequential
problems. The problem posing us is therefore whether the distribution of
chunks is inherently sequential, and we have already shown that it does
not have to be, with the conceptual possibility of the distribution of
chunks through space. Given this fact, a strong hint is given as to the
direction the distribution of chunks should take, and thereby the
composition of client and worker(s); at the very least, some degree of
parallelism will aid in efficiency. In a very concrete sense, the fact
of having all chunks in memory simultaneously should lead to orders of
magnitude faster interaction with them over having to deserialise them
from storage, due to the major speed advantages of computer memory with
respect to computer storage.

The secondary consideration which may cast doubt on this direction is
the practicality of distribution over multiple nodes. A parallel process
is no faster than a sequential process if there is only one single
processor, and is likely slower, due to the overhead. This is a
situationally variant aspect to consider, and can be enlightened by the
picture of the ideal audience for this project. The computational
statistician certainly has one computing node, so concurrent
distribution of chunks, over time, within the same node, is already a
given. If the statistician had access to multiple compute nodes, then
the question of practicality is already answered. Several scenarios
where access to multiple compute nodes are indeed conceivable:

\begin{itemize}
  \item
        Cloud computing cluster
  \item
        Office of unused computers
  \item
        Single machine with multi-core processor (each core treated as
        individual node)
\end{itemize}

As such, the considered audience would certainly have some practical
means to engage with chunks distributed spatially. Assuming a spatial
distribution of chunks, let us explore the effects on the composition of
processes within client, and within workers.

Upon issuing a request for operations to be run over some chunks, a
remarkable thing would occur on the client: nothing happens. The
processing occurs in an entirely separate memory space, with a different
processor. Were the client performing operations on regular, non-chunk
objects, the processing would bind up the client, and it would block
until the operation completed.

\cref{fig:work-request} shows a an example of this, where a client issues
a request for work, and is entirely free to do whatever other processing
is desired while the worker is blocked in managing the request.

\fig{work-request}{Communication diagram showing order of events for the requestof work}

The fact that the client is not blocked by operations on chunks opens up
a broad mix of possibilities and complications. In terms of
possibilities, the potential for chunk operations to be non-blocking to
the client means that operations may be \textbf{asynchronous}, which
would allow for significantly more efficient ordering of events within
the system. Long operations on chunks may take place side-by-side with
client-side operations on local objects, neither interfering with each
other. The converse of such a possibility is the new potential for
\textbf{race conditions}, where the unordered timing of events may lead
to undesirable behaviour. For example, if the client was waiting on the
result of operations that had a dependency between them that was never
fulfilled due to the disruption to their ordering, an algorithm may
never terminate. Race conditions can be worked around with cautious
programming backed by theory, but if the system had measures in place to
prevent many of the common errors that may arise with asynchrony, it
would be significantly more user-friendly as a result. Such measures may
include having some small degree of blocking in the client to return
acknowledgement of a valid request to a worker, or if experience shows
the race conditions to be insurmountable, disabling asynchrony by
default.

The emergence of a chunk on the client may also take several different
forms, and the system may very well possess the capability for all of
them. The key point to be made is that the existence of a chunk can not
be assumed across the system. The different knowledge of a chunk can be
relatively considered as different forms of existence of chunks, so it
is worth momentarily digressing to discuss them: Chunks may exist, or
they may not yet exist.

\begin{itemize}
  \item
        They exist when they are stored in memory on a worker.
  \item
        They do not yet exist when they are the awaited result of an operation
        on other chunks, or still to be read.
\end{itemize}

Both forms may have references to them; if a chunk reference is
immediately returned on the client from requesting a chunk operation,
the referenced chunk may not exist until a later time. Until such time
it does not exist, though the client has no knowledge of when it does
unless some special communications take place. This has ramifications
for the emergence of chunks to the client, which have to account for the
fact that a chunk that is requested to emerge, may not yet exist.

How may a potentially non-existent chunk be handled?

The straightforward, potentially naive strategy is to simply wait until
a chunk exists, upon issuing an emerge request from the client. The
client blocks until emergence, returns the result, and continues. An
example is given in a modified version of the previous diagram as \cref{fig:work-emergence}

\fig{work-emergence}{Communication diagram showing order of events for the emergence of work}

Were it not to block, and continue with program flow, the returned value
from the emerge would not correspond to the chunk value, and corruption
would ensue. However, blocking may lead to inefficiencies in execution,
so an alternative option is to allow for a different type of operation
that determines the availability of a particular chunk. A program may
then be constructed to take different paths based on the existence of a
chunk, with the potential to use the blocking emerge when the chunk is
available - which, given that it is available, wouldn't block any more
than is necessary over the details of the chunk transfer.

An example is given in \cref{fig:work-request-checking}

\fig{work-request-checking}{Communication diagram showing order of events for the emergence
    of work}

Further alternatives, such as promises, are described in greater detail
in the appendix.

The complement of chunk operations on the client is that of chunk
operations on the worker. The act of gathering the chunks takes on a new
difficulty when the chunks have varying levels of existence, scattered
over separate nodes, and similar issues as explored with the emerge on
the client are to be found. If a blocking emerge is issued on a worker
to pull a chunk from another worker, there is strong potential for race
conditions, especially if the two workers are seeking to pull from each
other, but both chunks don't yet exist. Blocking emerges are an
impossibility among workers, and they must all allow for some
concurrency in their emergences of chunks for their chunk operations.
This concurrency may also take different forms. It may be a matter of
issuing requests for chunks to be sent only when they are ready, and
returning to their main event loop, keeping a handle in memory of which
chunks relate to which operations.

This is depicted in diagram \cref{fig:argument-attain}:

\widefig{argument-attain}{Communication diagram showing manner of attaining arguments}

It may also be a case of checking on existence of all needed chunks in
an alternation with the main event loop, and emerging those that show as
existing - though this has the potential for high processor and network
utilisation if run too often, as well as greater inefficiences if run
too seldom. Irrespective, the implementation will have to take such a
factor for race conditions into consideration, lest the system lock up
and fail in its task. The other procedures relating to workers have less
of an essential consideration of concurrency in them: Running the chunk
operation can be performed in parallel to the main loop, as it is
independent, but it is not absolutely required. Likewise, the act of
storing the result of the chunk operation doesn't have the necessity of
concurrency attached to it, as it doesn't cause a worker to interact
directly with any of the other workers in the system.
