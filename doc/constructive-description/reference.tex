Through the defining feature of distributed data being incapable of direct reference within memory, we have established that there must be some entity through which to indirectly interact with the distributed data.
This entity we have given the appellation of \textbf{`Distributed Object Reference'}, and serves as a proxy which is directly affected by the user in order to pass on further information to and from the distributed data which the reference refers to.
The relationship between the reference and the distributed data is known as \textbf{``adequacy''} in formal semantics, with the role of the distributed data as the object of indirect interaction being called the \textbf{``referent''}.
Such a relationship is nothing new or groundbreaking; the very concept of language itself, with symbols ``standing for'' some intended entity, encapsulates this relationship by definition.
What we want to investigate in this section, is how such a relationship, which in many ways serves to \emph{define} interaction in our system, may be expressed, both semantically and syntactically.

First, it must be established that the reference itself may be of conceptual interest.
The referents themselves, if completely transparent to a reference, possess no information on their relation to each other.
This information needs to be captured somewhere accessible to the user, with most distributed algorithms being dependent on the knowledge of how the underlying chunks relate, in dimension or quantity.
For a specific example, see \cref{ch:lasso}.
Such information may be stored with each referent chunk, but it may also be argued that the metadata belongs with the reference, being the single entity representing many others, it fits well conceptually.
Were this to be the form of binding such information, the reference requires some means of direct access, rather than being a plain proxy to pass through.

To make the concept more concrete, let's explore some pre-existing examples embodying a reference relationship.
The most well-established example in computer science, one which Knuth refers to as a ``most valuable treasure'' alongside assignment operations, is the pointer.
A pointer serves as an object holding a specific memory address, with whatever value that address contains being the referent.
A pointer has the operation of \textbf{dereferencing} defined, whereby the referent value at the stored address is retrieved.
Here, there is a subtle additional piece of indirection, where a pointer references a memory address which references the value, but assuming the transitivity of reference, we can skip over such details.
The syntax for the declaration of a pointer in the \proglang{C} language takes the following form:

\code[c]{declare-int-pointer}

Where in this example, we have a pointer (itself the object of reference of the symbol \code[c]{p}), which points to memory holding some integer.
Assuming that this piece of memory exists and has a value, the pointer may be dereferenced using the \code[c]{star} operator as in \code[c]{deref-c}.

And we have assigned the value stored in the address referred to by \code{p}, to the variable \code{x}.
The symbol \code{x} now stands for that value.
The pointer itself may also be an object of interaction, with the C language allowing arithmetic on pointers in order to shift to a different location in memory.
This is important to note, as the pointer is not completely transparent - it is an object in its own right, something we will have to consider in greater detail when it comes to the references in our distributed system.

Let's look at a system closer to our use-case: \pkg{pbdR}.
Specifically, consider the distributed matrix class, \code{ddmatrix}, provided by the \pkg{pbdDMAT} package.
Here is a more complex reference than a pointer, where each \code{ddmatrix} reference may relate to an arbitrary number of matrix blocks distributed across potentially disparate memory space.
Operations over \code{ddmatrix} objects are directly reflected as operations on the referent blocks, as demonstrated in the \cref{lst:proxy-pbddmat}, where mathematical operations are passed on indirectly through the \code{ddmatrix} reference to the underlying matrix blocks:

\src{proxy-pbddmat}{Proxying of matrix operations in \pkg{pbdDMAT}}

A subtlety of \pkg{pbdR} is given in the point of view taken for the specification of code.
Rather than writing from the standpoint of a single location directing others, as a pointer is, \pkg{pbdR} code is written to be run identically across all nodes on the cluster.
The entire script is sent to all nodes, which run it in communication with each other.
A \code{ddmatrix} object thereby refers both to the local block held by the node executing the code, as well as all other blocks held by all other nodes.
For trivial operations such as the mathematical calculations given in the example above, there is no need to reference other blocks, with the \code{ddmatrix} object being used to refer solely to the local block.
In more complex operations, such as singular value decomposition, provided by \code{svd}, the reference given by \code{ddmatrix} will access all other blocks - with all of this being buried under a clean layer of abstraction.
For most cases, the referential nature of \code{ddmatrix} can be forgotten about, with the only adjustment required from the programmer migrating local code to distributed being the inversion of writing code for all nodes simultaneously.

The abstraction of reference can be equally cleanly broken, however, with the user capable of commanding the reference to engage only with the local block, through functions encoded with an \code{l} for ``local'' prefix.
The example given by \cref{lst:dim-ldim} demonstrates the difference in \pkg{pbdR} between \code{dim}, which gives the total dimensions of the entire matrix, and \code{ldim}, which gives only the dimensions of the block local to a particular node:

\src{dim-ldim}{The difference between dim and ldim in \pkg{pbdDMAT}}

In general, \code{submatrix} returns the local storage for the \code{ddmatrix} local referent.
We have with \pkg{pbdR} a converse of pointer syntax: the default with \pkg{pbdR} is to pass through to the referent, with special syntax required to access the local reference.

With these examples considered, let's abstract somewhat, and then see what form reference relations may take in our system.
The reference relation gains greater clarity when considered in the light of what is known in analytic philosophy as the \textbf{``Use-Mention Distinction''}\cite{quine2009}.
The \emph{use} of a term is where the term is used as part of a statement not relating to the term itself.
For example, a use of the term ``matrix'', may be in the sentence, ``a block matrix is a matrix in which sections may be partitioned into blocks''.
The \emph{mention} of a term is where the term as a signifier itself is used as a subject of the statement.
For example, a mention of the term ``matrix'', may be in the sentence, ``the word `matrix' may be partitioned into 6 letters''.
The relevance of the use-mention distinction to distributed objects lies in the contrasting operations that may be intended of a distributed object reference: interaction may be intended for either the referent distributed object (use), or the reference itself (mention).
Such semantics of access to the reference and referent serve as the basis for much of meta-programming, with the \proglang{lisp} quotation and quasi-quotation enabling explicit use/mention interaction.

\proglang{Lisp} possesses an extremely simple syntax, with everything in brackets evaluating according to the first element as the operator and the following elements as operands.
For example, \code[lisp]{sum} evaluates to \code[lisp]{6}, as summation.
Values also self-evaluate; \code[lisp]{6} evaluates to \code[lisp]{6}.
The evaluation treats everything as ``in-use''.
Such behaviour can be prevented through quotation constructs.
For example, \code[lisp]{quote-sum} evaluates to \code[lisp]{sum}, which is just treated as just a list of symbols (mention), ready for evaluation later on.
Such a construct is so common, it has the abbreviated form of the \code[lisp]{tick} - the above example may have been equivalently quoted using \code[lisp]{tick-sum}. Quotation may also act in a more selective manner, where some symbols are used, and others just mentioned, with the ``back-quote'' operator of quasi-quotation.
Here, \code[lisp]{backquote} is used to quote, with \code[lisp]{quasiquote} used to unquote.
For example, \code[lisp]{quoted} evaluates to \code[lisp]{unquoted}.

This elegant back-and-forth switching between use and mention of the symbolic reference is a cornerstone of the language.
The switching of use and mention of the distributed object reference would serve to benefit from such semantics and syntax.

With this in place, we might now consider what semantics for reference and referent are optimal with respect to distributed object reference and distributed object.

First, it is clear from the example of \pkg{pbdR} that having the reference as a mostly-transparent proxy yields very clear code.
It seems a reasonable starting point to base defaults on.
With passing through most operations to the referent through the reference, there is minimal overhead in transferring code from local-only to distributed, which fits in well with the ideal of having the code as close to mathematical notation as possible.
Potential issues that arise from this include accidental calls to the distributed object when the reference is intended.
With more interaction with the reference than referent however, this would cause less damage than the converse.

What form might this take, ideally?
There are two components: interaction with the referent, and interaction with the reference.
Using \R{} semantics, the ideal for interaction with the referent would be some wildcard generic which wrapped all possible functions, with a method for distributed object references that captured the calling function and sent it directly through to the referent, dereferencing when necessary.
This would guarantee all functions to be passed through the reference as a proxy, including the nonsensical - a higher layer of abstraction may be written to filter functionality.
Reference interaction would benefit from some special operator indicating that the reference \emph{per se} is the intended object of interaction.
We term such an operation as \textbf{``materialisation''} of the reference, and further operations occur directly to the reference until it is again returned to serving as a proxy through some \textbf{``dematerialisation''} This special operator gets around the need to create special-case functions for access that break the consistency of passing through references as proxy objects.
Proximal to the \code[c]{star} operator for the dereferencing operation in \proglang{C} is the \code[c]{address} operator for accessing the address of a value, going in the opposite direction.
This analogy is somewhat imperfect however, as in \proglang{C}, interaction is with the reference by default.
While the semantics don't map exactly with the concepts of address and location, something akin to the operators may be desirable.

It may also be asked at what level of abstraction such distinctions may be made.
The constant need to materialise chunks at the developer level is likely to lead to bugs of forgotten materialisations, so the default may best be set to not allow chunk proxying unless explicitly invoked.

From a practical standpoint, most programming languages in current use don't possess the capacity for the ideal forms of interaction described above.
Practically speaking, distributed object references must be interacted with using functions and methods.
Referent interaction may take place through wrapping the relevant functions manually, perhaps through some higher-order wrapper provided by the system, that may output a function which proxies the distributed object reference.
Common methods may have this already provided.
Reference interactions may take several different forms.
Like \pkg{pbdR}, they may have some special encoding in function names, as in \code{ddim}.
This is similar to C's \texttt{\#}-directive pre-processing, though it requires an additional modest burden on the developer.
Alternatively, functions may be provided to \code{materialise} and dematerialise the distributed object reference, which serve as the only special cases to access the reference directly.

The relation between distributed object reference and the referent distributed object will always be in some tension, though hopefully through careful consideration as in these sections such a tension may be eased to the point of being forgotten about by the user, who can ignore distinctions of use and mention until absolutely necessary, and even then having a straightforward experience free of surprise or confusion.
