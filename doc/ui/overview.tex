\Cref{ch:initial-experiments} described the first two iterations of development in \lsr{}.
The second iteration served to improve on the first in many ways, not least through the gains in asynchronous processing.
The structure remains permanently limited however through the dependence on a central message queue, as a bottleneck.
However, many aspects clearly worked, in the small demonstrations of expressivity and capability for performing statistical calculations.
Given the successes, coupled with the growing complexity of the final prototype discussed in \cref{sec:mq}, it is at this point worthwhile to consolidate the components which have proven to be essential in the system, and factor them into a cohesive system for final delivery.

This chapter presents the User Interface for the consolidated system, in four measures; first, a description of the modules composing the API, given by \cref{sec:sysmod}.
This is followed by an example-led tour of the core functions offered by the main system module in \cref{sec:api-example}.
\Cref{sec:api-high,sec:api-low} round out the remainder of the high-level modules and low-level modules, respectively.
