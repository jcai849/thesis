\hypertarget{introduction}{%
\section{Introduction}\label{introduction}}

Given the simplicity and promise of flexibility as demonstrated in the
documents \href{inter-node-comm-w-redis.html}{Inter-node Communication
with Redis} and \href{message-queues-comms.html}{Message Queue
Communications}, further experimentation around the concept is
undertaken and documented herein. The experiments are built successively
upon it's prior, with the aim of rapidly approximating a functioning
prototype via experimentation.

\hypertarget{general-function-on-single-chunk}{%
\section{General Function on Single
Chunk}\label{general-function-on-single-chunk}}

While the RPC-based architecture as described in
\href{experiment-eager-dist-obj-pre.html}{Experiment: Eager Distributed
Object} had significant limitations, a particularly powerful construct
was the higher order function \texttt{distributed.do.call}, which took
functions as arguments to be performed on the distributed chunks.

This construct is powerful in that it can serve as the basis for nearly
every function on distributed chunks, and this section serves to
document experiments relating to the creation of a general function that
will perform a function at the node hosting a particular chunk.

\hypertarget{sec:val-ret}{%
\subsection{With Value Return}\label{sec:val-ret}}

Regardless of performing the actual function, some means of returning
the value of a function must be provided; this section focuses on
getting a function to be performed on a server node, with the result
send back to the client. Listings of an implementation of these concepts
are given by \cite{lst:vr-client}\cite{lst:vr-server}.

\hypertarget{lst:vr-client}{%
\label{lst:vr-client}}%
\begin{verbatim}
#!/usr/bin/env R

library(rediscc)

RSC <- redis.connect(host="localhost", port=6379L)
SELF_ADDR <- list(host="localhost", port=12345L)
chunk <- structure("chunk1", class = "chunk")

main <- function() {
    doFunAt(fun=exp, chunk=chunk)
}

doFunAt <- function(fun, chunk, conn) {
    msg <- bquote(list(fun=.(fun),
               chunk=.(chunk),
               returnAddr=.(SELF_ADDR)))
    writeMsg(msg, chunk)
    cat("wrote message: ", format(msg), " to ", chunk, "\n")
    listenReply()
}

listenReply <- function() {
    replySock <- socketConnection(getHost(), getPort(), server=TRUE)
    response <- character(0)
    while (length(response) < 1) {
        response <- tryCatch(unserialize(replySock),
                     error = function(e) {
        cat("no reply, trying again in 1 sec\n")
        Sys.sleep(1); NULL})
    }
    cat("received response: ", format(response), "\n")
    close(replySock)
    response
}

getHost <- function() SELF_ADDR$host
getPort <- function() SELF_ADDR$port

writeMsg <- function(msg, to) {
    serializedMsg <- rawToChar(serialize(msg, NULL, T))
    redis.push(RSC, to, serializedMsg)
}

main()
\end{verbatim}

\hypertarget{lst:vr-server}{%
\label{lst:vr-server}}%
\begin{verbatim}
#!/usr/bin/env R

library(rediscc)

RSC <- redis.connect(host="localhost", port=6379L)
chunk1 <- seq(10)
QUEUE <- "chunk1"

main <- function() {
    while (TRUE) {
        msg <- readMessage(QUEUE)
        cat("read message:", format(msg), "\n")
        result <- doFun(msg)
        cat("result is: ", format(result), "\n")
        reply(result, getReturnAddr(msg))
    }
}

doFun <- function(msg) {
    fun <- getFun(msg); arg <- getArg(msg)
    do.call(fun, list(arg))
}

reply <- function(result, returnAddr) {
    replySock <- NULL
    while (is.null(replySock))
        replySock <- tryCatch(socketConnection(getHost(returnAddr),
                               getPort(returnAddr)),
                  error = function(e) {
                      cat("Failed to connect to return address",
                      ", trying again..\n")
                      NULL})
    cat("replying to request...\n")
    serialize(result, replySock)
    cat("replied\n")
    close(replySock)
}

getFun <- function(msg) msg$fun
getArg <- function(msg) get(msg$chunk)
getReturnAddr <- function(msg) msg$returnAddr
getHost <- function(addr) addr$host
getPort <- function(addr) addr$port

readMessage <- function(queues) {
    serializedMsg <- redis.pop(RSC, queues, timeout=Inf)
    unserialize(charToRaw(serializedMsg))
}

main()
\end{verbatim}

To this end, the client node has a function defined as
\texttt{doFunAt(fun,\ chunk)}, which takes in any function, and the ID
of a chunk to perform the function on. An implementation is given by
\cite{lst:vr-client}. \texttt{doFunAt()} first composes a message to
send to the chunk's queue, being a list consisting of the function, the
chunk name, and a return address, which contains sufficient information
for the node performing the operation on the chunk to send the results
back to via socket connection. The message is then serialised and pushed
to the chunk's queue, and the requesting node sits listening on the
socket that it has set up and advertised.

On the server node end, it sits waiting on it's preassigned queues, each
of which correspond to a chunk that it holds. Upon a message coming
through, it runs a \texttt{doFun()} function on the message, which in
turn runs the function on the chunk named in the message. An
implementation is given by \cite{lst:vr-server}. It then creates a
socket connected to the clients location as advertised in the message,
and sends the serialised results through.

A problem with this approach is the fickle aspect of creating and
removing sockets for every request; beyond the probability of missed
connections and high downtime due to client waiting on a response, R
only has a very limited number of connections available to it, so it is
impossible to scale beyond that limit.

\hypertarget{with-assignment}{%
\subsection{With Assignment}\label{with-assignment}}

Assigning the results of distributed operation to a new chunk is a far
more common operation in a distributed system in order to minimise data
movement. This will involve specifying additional directions as part of
the request message, in order to specify that assignment, and not merely
the operation, is desired.

It will be clear from the previous example that the problem of
point-to-point data movement, somewhat solved via direct sockets in that
previous example, is largely an implementation issue, and a problem
entirely distinct to the remainder of the logic of the system. From this
experiment onwards, the mechanism of data movement is abstracted out,
with the assumption that there will exist some additional tool that can
serve as a sufficient backend for data movement. In reality, until that
tool is developed, data will be sent through redis; not a solution, but
something that can be ignored without loss of generality.

The actual creation of a chunk ID in itself demands a system-wide unique
identifier; this is a solved problem with a central message server, in
redis providing an \texttt{INCR} operation, which can be used to
generate a new chunk ID that is globally unique.

The name origination and option of blocking until a chunk is formed will
dictate different algorithms in the creation of the distributed chunk
object, as well as the structure of the distributed chunk object.
\cite{tbl:name-orig-block} shows potential forms these may take. In
addition, the ``jobID'' referred to in the table may take the concrete
form of a simple key-value store, with the key being passed and
monitored by the client node.

\begin{longtable}[]{@{}
  >{\raggedright\arraybackslash}p{(\columnwidth - 4\tabcolsep) * \real{0.0462}}
  >{\raggedright\arraybackslash}p{(\columnwidth - 4\tabcolsep) * \real{0.3250}}
  >{\raggedright\arraybackslash}p{(\columnwidth - 4\tabcolsep) * \real{0.6288}}@{}}
\caption{Description of Algorithms and Data Structure of chunk reference
object, by blocking status in creation, and origination of chunk ID.
\{\#tbl:name-orig-block\}}\tabularnewline
\toprule()
\begin{minipage}[b]{\linewidth}\raggedright
\end{minipage} & \begin{minipage}[b]{\linewidth}\raggedright
Client-Originated chunk ID
\end{minipage} & \begin{minipage}[b]{\linewidth}\raggedright
Server-Originated chunk ID
\end{minipage} \\
\midrule()
\endfirsthead
\toprule()
\begin{minipage}[b]{\linewidth}\raggedright
\end{minipage} & \begin{minipage}[b]{\linewidth}\raggedright
Client-Originated chunk ID
\end{minipage} & \begin{minipage}[b]{\linewidth}\raggedright
Server-Originated chunk ID
\end{minipage} \\
\midrule()
\endhead
Blocking Algorithm & client attains chunk ID, sends operation request
withchunk ID to server, creating chunk reference concurrently, blocking
until direct signal of completion from server. & client sends operation
request with reference to some common information repository and the job
ID to server. server attains chunk ID, performs operation, and sends
chunk chunk ID to thejob ID at the common information repository, which
client watches, releasing chunk object after attaining chunk ID from
repository. \\
Blocking Structure & String name of chunk. & String name of chunk. \\
Non-Blocking Algorithm & client attains chunk ID, sends operation
request withchunk ID to server, creating chunk reference concurrently.
Nowaiting for server signal of completion. & client sends operation
request with reference to some common information repository and the job
ID to server. server attains chunk ID, performs operation, and sends
chunk ID to the job ID at common information repository. Before server
completion, client releases chunk object, not waiting for reception of
chunk information. \\
Non-Blocking Structure & String name of chunk & Initially, reference to
common information repository. Mutable; can become string name of chunk
upon accessing that information in the common information repository. \\
\bottomrule()
\end{longtable}

While it is clearly more straightforward for a client node to originate
a chunk ID, with blocking, the opposite will possibly be the most
flexible; server-originated chunk ID with no blocking. This is because
the very existence of a chunk is presupposed when a client node
originates a chunk ID, while that may not be true in reality. For
instance, the result may be an unexpected \texttt{NULL}, zero-length
vector, or even an error. In addition, the server-originated chunk ID
with no blocking has every feature common to that of a future, from the
future package; it can be checked for completion, and accessed as a
value, allowing for many asynchronous and parallel operations.

\hypertarget{client-originated-chunk-id}{%
\subsubsection{Client-Originated Chunk
ID}\label{client-originated-chunk-id}}

The logic of the client in assigning the result of a distributed
operation on a chunk is largely encapsulated in a new function,
\texttt{assignFunAt()}, as demonstrated in \cite{lst:ro-ass-client}. The
function attains a chunk ID, generates a unique return address, sends a
message to the operand chunk queue, and waits for a reply, before
returning the id as a string belonging to the ``chunk'' class. There is
more information in the message relative the the function-only message
of section \cite{sec:val-ret}; the chunk ID, request for acknowledgement
of completion, return address, as well as an operation specifier to
direct the intent of the message.

\hypertarget{lst:ro-ass-client}{%
\label{lst:ro-ass-client}}%
\begin{verbatim}
#!/usr/bin/env R

library(rediscc)
library(uuid)

RSC <- redis.connect(host="localhost", port=6379L)
redism(RSC, "chunkID")
chunk1 <- structure("chunk1", class = "chunk")
redism(RSC, "chunk1")
redism(RSC, as.character(1:10))

main <- function() {
    x = assignFunAt(fun=expm1, chunk=chunk1, wait=F)
    y = assignFunAt(fun=log1p, chunk=x, wait=T)
}

assignFunAt <- function(fun, chunk, wait=TRUE) {
    id <- getChunkID()}
    returnAddr <- UUIDgenerate()
    sendMsg("ASSIGN", fun, chunk, returnAddr, id, ack = wait)
    if (wait) readReply(returnAddr)
    structure(id, class = "chunk")
}

doFunAt <- function(fun, chunk) {
    returnAddr <- UUIDgenerate()
    sendMsg("DOFUN", fun, chunk, returnAddr)
    readReply(returnAddr)
}

getChunkID <- function() as.character(redis.inc(RSC, "chunkID"))

readReply <- function(addr, clear=TRUE) {
    reply <- redis.pop(RSC, addr, timeout = Inf)}
    if (clear) redism(RSC, addr)
    unserialize(charToRaw(reply))
}

sendMsg <- function(op, fun, chunk, returnAddr, id=NULL, ack=NULL) {
    msg <- newMsg(op, fun, chunk, id, ack, returnAddr)
    writeMsg(msg, chunk)
}

newMsg <- function(op, fun, chunk, id, ack, returnAddr) {
    structure(list(op = op, fun = fun, chunk = chunk,
               id = id, ack = ack, returnAddr = returnAddr),
          class = "msg")
}

writeMsg <- function(msg, to) {
    serializedMsg <- rawToChar(serialize(msg, NULL, T))
    redis.push(RSC, to, serializedMsg)
}

format.chunk <- function(x, ...) {
    obj <- doFunAt(identity, x)
    format(obj)
}

main()
\end{verbatim}

The server, as shown in \cite{lst:ro-ass-server}, consists in a loop of
reading the message and performing an operation dependent on the
operation specifier of the message. For an operation of \texttt{DOFUN},
all that is run is a \texttt{do.call()} on the function and chunk
specified, with a message being returned to the client with the value of
the \texttt{do.call()}. An operation of \texttt{ASSIGN} runs the same as
\texttt{DOFUN}, with the addition of assigning the value to the ID as
passed in the message, adding the ID to the array of queues to monitor,
and potentially sending acknowledgement back to the client node.

\hypertarget{lst:ro-ass-server}{%
\label{lst:ro-ass-server}}%
\begin{verbatim}
#!/usr/bin/env R

library(rediscc)

RSC <- redis.connect(host="localhost", port=6379L)
chunk1 <- seq(10)
QUEUE <- "chunk1"

main <- function() {
    while (TRUE) {
        msg <- readMessage(QUEUE)
        cat("read message:", format(msg), "\n")
        switch(getOp(msg),
               "ASSIGN" = {assignFun(getFun(msg), getChunk(msg),
                         getChunkID(msg))
                   if (getAck(msg))
                     writeMsg("Complete", getReturnAddr(msg))},
               "DOFUN" = writeMsg(doFun(getFun(msg), getChunk(msg)),
                      getReturnAddr(msg)))
    }
}

assignFun <- function(fun, chunk, id) {
    val <- doFun(fun, chunk)
    assign(id, val, envir = .GlobalEnv)
    assign("QUEUE", c(QUEUE, id), envir = .GlobalEnv)
}

doFun <- function(fun, chunk) {
    do.call(fun, list(chunk))
}

getMsgField <- function(field) function(msg) msg[[field]]
getOp <- getMsgField("op"); getFun <- getMsgField("fun")
getChunk <- function(msg) get(getMsgField("chunk")(msg))
getChunkID <- getMsgField("id"); getAck <- getMsgField("ack")
getReturnAddr <- getMsgField("returnAddr")

readMessage <- function(queues) {
    serializedMsg <- redis.pop(RSC, queues, timeout=Inf)
    unserialize(charToRaw(serializedMsg))
}

writeMsg <- function(msg, to) {
    serializedMsg <- rawToChar(serialize(msg, NULL, T))
    redis.push(RSC, to, serializedMsg)
    cat("wrote message: ", format(msg),
        " to queue belonging to chunk \"", to, "\"\n")
}

main()
\end{verbatim}

\hypertarget{server-originated-chunk-id}{%
\subsubsection{Server-Originated chunk
ID}\label{server-originated-chunk-id}}

By this point the client (\cite{lst:wo-ass-client}) and server
(\cite{lst:wo-ass-server}) come to increasingly resemble each other, and
most of the functions are shared, as in listings \cite{lst:wo-ass-chunk}
\cite{lst:wo-ass-shared}\cite{lst:wo-ass-messages}.

The principal mechanism of action is best demonstrated via a logical
time diagram, given by figure (missing), following a Lamport form of
event ordering \cite{lamport1978ordering}. The first message, shown by
the \textbf{a} arrow in the diagram, involves a client sending a message
to a server regarding the request, including the job ID naming a queue
in a shared information reference for the server to later place the
chunk ID into.

Optionally, the client can immediately create a chunk object with no
direct knowledge of the chunk ID, holding the job ID at the information
reference instead, and the client continues whatever work it was doing.
Only when the chunk ID is required, the chunk object, triggers a
blocking pop on it's associated information reference queue, which the
server may at any point push the chunk ID to. The chunk object then has
the associate the ID associated with it, and the information reference
queue can be deleted.

\hypertarget{lst:wo-ass-client}{%
\label{lst:wo-ass-client}}%
\begin{verbatim}
#!/usr/bin/env R

source("shared")
source("messages")
source("chunk")

distInit()
rediscc::redism(conn(), c("distChunk1", paste0("C", 1:10), paste0("J", 1:10),
        "JOB_ID", "CHUNK_ID"))
distChunk1 <- structure(new.env(), class = "distChunk")
chunkID(distChunk1) <- "distChunk1"

main <- function() {
    cat("Value of distChunk1:", format(distChunk1), "\n")
    x <- do.call.distChunk(what=expm1, chunkArg=distChunk1,
                   assign=T, wait=F)
    cat("Value of x:", format(x), "\n")
    y <- do.call.distChunk(log1p, x, assign=T, wait=T)
    cat("Value of y:", format(y), "\n")
}

main()
\end{verbatim}

\hypertarget{lst:wo-ass-server}{%
\label{lst:wo-ass-server}}%
\begin{verbatim}
#!/usr/bin/env R

source("shared")
source("messages")
source("chunk")

distInit()
distChunk1 <- seq(10)
QUEUE <- "distChunk1"

main <- function() {
    repeat {
        m <- read.queue(QUEUE)
        switch(op(m),
               "ASSIGN" = {cID <- do.call.chunk(what=fun(m),
                            chunkArg=chunk(m),
                            distArgs=dist(m),
                            staticArgs=static(m),
                            assign=TRUE)
                       send(CHUNK_ID = cID, to = jobID(m))},
               "DOFUN" = {v <- do.call.chunk(what=fun(m),
                             chunkArg=chunk(m),
                             distArgs=dist(m),
                             staticArgs=static(m),
                             assign=FALSE)
                      send(VAL = v, to = jobID(m))})
    }
}

do.call.chunk <- function(what, chunkArg, distArgs, staticArgs, assign=TRUE) {
    if (assign) {
        cID <- chunkID()
        v <- do.call(what, list(chunkArg))
        cat("Assigning value", format(v), "to identifier",
            format(cID), "\n")
        assign(cID, v, envir = .GlobalEnv)
        assign("QUEUE", c(QUEUE, cID), envir = .GlobalEnv)
        return(cID)
    } else do.call(what, list(chunkArg))
}

main()
\end{verbatim}

\hypertarget{lst:wo-ass-chunk}{%
\label{lst:wo-ass-chunk}}%
\begin{verbatim}
# distChunk methods

jobID.distChunk <- function(x, ...) get("JOB_ID", x)

chunkID.distChunk <- function(x, ...) {
    if (! exists("CHUNK_ID", x)) {
        jID <- jobID(x)
        cat("chunkID not yet associated with distChunk; checking jobID",
            jID, "\n")
        cID <- chunkID(read.queue(jID, clear=TRUE))
        cat("chunkID \"", format(cID), "\" found; associating...\n",
            sep="")
        chunkID(x) <- cID
    }
    get("CHUNK_ID", x)
}

do.call.distChunk <- function(what, chunkArg, distArgs=NULL, staticArgs=NULL,
                  assign=TRUE, wait=FALSE) {
    jID <- jobID()
    cat("Requesting to perform function", format(what), "on chunk",
        chunkID(chunkArg), "with",
        if (assign) "assignment" else "no assignment", "\n")
    send(OP = if (assign) "ASSIGN" else "DOFUN", FUN = what,
         CHUNK = chunkArg, DIST_ARGS = distArgs, STATIC_ARGS = staticArgs,
         JOB_ID = jID, to = chunkID(chunkArg))

    dc <- if (assign) {
        if (!wait){
            cat("not waiting, using job ID", format(jID), "\n")
            distChunk(jID)
        } else {
            distChunk(chunkID(read.queue(jID, clear=TRUE)))
    } } else {
        val(read.queue(jID, clear=TRUE))
    }
    dc
}

format.distChunk <- function(x, ...) {
    c <- do.call.distChunk(identity, x, assign=FALSE)
    format(c)
}
\end{verbatim}

\hypertarget{lst:wo-ass-shared}{%
\label{lst:wo-ass-shared}}%
\begin{verbatim}
# Generics and setters

distChunk <- function(x, ...) {
    if (missing(x)) {
        dc <- new.env()
        class(dc) <- "distChunk"
        return(dc)
    }
    UseMethod("distChunk", x)
}

chunkID <- function(x, ...) {
    if (missing(x)) {
        cID <- paste0("C", rediscc::redis.inc(conn(), "CHUNK_ID"))
        class(cID) <- "chunkID"
        return(cID)
    }
    UseMethod("chunkID", x)
}

`chunkID<-` <- function(x, value) {
    assign("CHUNK_ID", value, x)
    x
}

jobID <- function(x, ...) {
    if (missing(x)) {
        jID <- paste0("J", rediscc::redis.inc(conn(), "JOB_ID"))
        class(jID) <- "jobID"
        return(jID)
    }
    UseMethod("jobID", x)
}

`jobID<-` <- function(x, value) {
    assign("JOB_ID", value, x)
    x
}

chunk <- function(x, ...) UseMethod("chunk", x)

dist <- function(x, ...) UseMethod("dist", x)

dist.default <- stats::dist

# jobID methods

distChunk.jobID <- function(x, ...) {
    dc <- distChunk()
    jobID(dc) <- x
    dc
}

# chunkID methods

distChunk.chunkID <- function(x, ...) {
    dc <- distChunk()
    chunkID(dc) <- x
    dc
}

# Initialisation

init <- local({
    rsc <- NULL

    distInit <- function(queueHost="localhost", queuePort=6379L, ...) {
        # Place for starting up worker nodes
        rsc <<- rediscc::redis.connect(queueHost, queuePort)
    }

    conn <- function() {
        if (is.null(rsc))
            stop("Redis connection not found. Use `distInit` to initialise.")
        rsc
    }
})

distInit <- get("distInit", environment(init))
conn <- get("conn", environment(init))
\end{verbatim}

\hypertarget{lst:wo-ass-messages}{%
\label{lst:wo-ass-messages}}%
\begin{verbatim}
# messaging functions

msg <- function(...) {
    structure(list(...), class = "msg")
}

send <- function(..., to) {
    items <- list(...)
    m <- do.call(msg, items)
    write.msg(m, to)
}

write.msg <- function(m, to) {
    serializedMsg <- rawToChar(serialize(m, NULL, T))
    rediscc::redis.push(conn(), to, serializedMsg)
    cat("wrote message: ", format(m),
        " to queue belonging to chunk \"", to, "\"\n", sep="")
}

read.queue <- function(queue, clear = FALSE) {
    cat("Awaiting message on queues: ", format(queue),  "\n", sep="")
    serializedMsg <- rediscc::redis.pop(conn(), queue, timeout=Inf)
    if (clear) rediscc::redism(conn(), queue)
    m <- unserialize(charToRaw(serializedMsg))
    cat("Received message:", format(m), "\n")
    m
}

# message field accessors

msgField <- function(field) function(x, ...) x[[field]]
# Requesters
op <- msgField("OP"); fun <- msgField("FUN")
static <- msgField("STATIC_ARGS")
chunk.msg <- function(x, ...) get(chunkID(msgField("CHUNK")(x)))
jobID.msg <- msgField("JOB_ID"); dist.msg <- msgField("DIST_ARGS")
# Responders
val <- msgField("VAL"); chunkID.msg <- msgField("CHUNK_ID")
\end{verbatim}
