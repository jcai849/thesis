\subsection{Bigmemory Collection}\label{subsec:bigmemory-collection}

The \pkg{bigmemory} package enables the creation of ``massive matrices''
through a ``big.matrix'' S4 class with a similar interface to
`matrix'\cite{kane13:bigmemory}. These matrices may take up gigabytes of
memory, typically larger than RAM, and have simple operations defined
that speed up their usage. A variety of extension packages are also
available that provide more functionality for big.matrices. The massive
capacity of big.matrices is given through their default memory
allocation to shared memory, rather than working memory as in most R
objects. The objects in \R{} are therefore pointers, and the big.matrix
``show'' method prints a description and memory location instead of a
standard matrix display, given that it is likely far too big a matrix to
print reasonably. Parallel processing is made use of for the advantage
of computations and subsetting of matrices. Development on the package
is still active, however it is stable enough that updates are
intermittent. Some of the main extension packages:

\begin{description}
    \item[\pkg{biganalytics}]
        Extends \pkg{bigmemory} with matrix summary statistics such as
        \code{colmeans}, \code{apply}, as well as integration with the \pkg{biglm}
        package\cite{emerson16}. \pkg{biganalytics} is authored by the same creators
        of the main \pkg{bigmemory} package.
    \item[\pkg{bigtabulate}]
        Extends \pkg{bigmemory} with tabulation functions and \code{tapply},
        allowing for contingency tables and \code{summary} of big.matrix
        objects \cite{kane16}.
    \item[\pkg{biglasso}]
        Extends \pkg{bigmemory} matrices to allow for lasso, ridge and elastic-net
        model fitting. It can be take advantage of multicore machines, with the
        ability to be run in parallel. \pkg{biglasso} is authored by Yaohui Zeng, and
        described in detail in \textcite{zeng2017biglasso}.
    \item[\pkg{bigalgebra}]
        Provides \pkg{BLAS} routines for \pkg{bigmemory} and native \R{} matrices. Linear
        Algebra functionality is given through matrix arithmetic methods, such
        as the standard \code{matrix-multiply}. The package is archived
        on CRAN as of February 2020, only being accessible through R-Forge. This
        is likely due to a merger with the main \pkg{bigmemory} package.
    \item[\pkg{bigstatsr}]
        Was originally a set of functions for complex statistical analyses on
        big.matrices, having since implemented and provided it's own
        ``file-backed big matrices''\cite{prive2018efficient}. The provided
        functions include matrix operations particularly relating to
        bioinformatics, such as PCA, sparse linear supervised models, etc.
        \pkg{bigstatsr} is described in detail in \textcite{prive2018efficient}.
\end{description}

\subsubsection{LAPACK, BLAS, ATLAS}\label{subsec:blas-lapack}

\pkg{BLAS} is a specification for a set of low-level ``building block'' linear
algebra routines\cite{lawson1979basic}. Most linear algebra libraries
conform to the \pkg{BLAS} specifications, including the most prominent linear
algebra library, \pkg{LAPACK}, with it's own set of
extensions\cite{demmel1989lapack}. \pkg{LAPACK} has been extended in turn to
support a variety of systems, with implementations such as \pkg{ScaLAPACK}
being introduced to attend to distributed memory
systems\cite{choi1992scalapack}.

\subsection{disk.frame}\label{subsec:disk.frame}

\pkg{disk.frame} provides a disk.frame class and derivatives, which model a
data.frame. The primary functional difference is that disk.frames can be
far larger than total RAM. This is enabled through the disk.frame
objects being allocated to shared memory, rather than working memory as
in data.frames. The transparency offered by the class is well-known to
be at a very high level, with most standard manipulations of dataframes
being applicable to disk.frame objects. \pkg{disk.frame} is expanded upon in \cref{sec:disk-frame-study}.

\subsection{data.table}\label{subsec:data.table}

data.table is another dataframe alternative, focussing on speed through
multithreading and well-tuned database algorithms\cite{dowle19}. The
package has introduced a unique syntax for data.table manipulation,
which is also made available in \pkg{disk.frame}. data.table objects are held
in RAM, so big data processing is not easily enabled, however the
package allows for serialisation of data.tables, and chunking is
possible through splitting via the \code{shard} function. The package
is authored by Matt Dowle, currently an employee at H2O.ai. An overview
is given in \cite{dowle19:_introd}, with extensive benchmarking
available in \cite{dowle19:_bench}.

\subsection{fst}\label{sec:fst}

\pkg{fst} is a means of serialising dataframes, as an alternative to RDS
files\cite{klik19}. Serialisation takes place extremely fast, using
compression to minimise disk usage. The package speed is increased
through parallel computation. Author Mark Klik and Yann Collet, of
Facebook, Inc. \pkg{fst} is a dependency of \pkg{disk.frame}, performing some of the
background functionality.

\subsection{iotools}\label{subsec:iotools}

\pkg{iotools} is a set of tools for managing big I/O, with an emphasis on
speed and efficiency for big data through chunking\cite{urbanek20b}. The
package provides several functions for creating and manipulating chunks
directly. Authored by Simon Urbanek and Taylor Arnold.

\subsection{ff}\label{subsec:ff}

The package description outlines \pkg{ff} with the following:

\qu{adler18}

The package provides a disk-based storage for most base types in \R{}. This
also enables sharing of objects between different \R{} processes. \pkg{ff} is
authored by a German-based team, and maintained by Jens Oehlschlägel,
the author of True Cluster. First introduced in
2008\cite{adler08:_large_r}, there have been no updates since
mid-2018.

\begin{description}
\item[\pkg{ffbase}\cite{jonge20}]
        is an extension to \pkg{ff}, providing standard statistical methods for ff
        objects. The package is still actively maintained. The package has been
        the subject of several talks, most notably the author's overview,
        \cite{wijffels13}. The package is currently being revised for a second
        version that provides generics functionality for \pkg{dplyr} on ff objects,
        under the title, \pkg{ffbase2}\cite{jonge15}.
\end{description}
