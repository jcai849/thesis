\hypertarget{sec:partools}{%
    \subsection{partools}\label{sec:partools}}

partools provides utilities for the parallel
package\cite{matloff16softw_alchemy}. It offers functions to split
files and process the splits across nodes provided by parallel, along
with bespoke statistical functions.

It consists mainly of wrapper functions, designed to follow it's
philosophy of ``keep it distributed''.

It is authored by Norm Matloff, a professor at UC, Davis and current
Editor-in-Chief of the R Journal.

In more detail, \cite{matloff15} and \cite{matloff17} presents the
motivation behind partools with reference to Hadoop and Spark. Matloff
describes partools as more ``sensible'' for large data sets than Hadoop
and Spark, due to their difficulty of setup, abstract programming
paradigms, and the overhead caused by their fault tolerance. The
alternative approach favoured by partools, termed ``software alchemy'',
is to use base R to split the data into distributed chunks, run analyses
on each chunk, then average the results. This is proven to have
asymptotic equivalence to standard analyses under certain assumptions,
such as iid data. Effectively, it is a map-reduce, with map being some
analysis, and reduce being an average.

The analyses amenable to software alchemy have bespoke functions for
them in the package, typically consisting of their base R name with the
prefix ``ca'' alluding to ``chunk averaging'', such as
\mintinline{r}{calm()}. Other functions in which it
doesn't make sense to average are also supported, such as column sums,
which also have specific functions made for them. Complex cases such as
fitting LASSO models, in which each chunk may have settled on different
explanatory variables, are catered for in partools through subsetting
them. Finally, aggregate functions, akin to R's aggregate function,
provide for arbitrary functions to be applied to distributed data.

In terms of applications of the package, it is difficult to estimate the
usage of it; as it has a more complex setup than a simple
\mintinline{r}{library} call, it won't be included in many other packages.
Similarly, the nature of the work skews towards interactive usage, and
custom business-specific programs that are difficult to attain data on.
The reverse dependencies/imports have all been authored by Matloff, so
aren't entirely informative, but their usage is interesting: one package
(cdparcoord) to plot coordinates for large datasets in parallel, and one
(polyreg) to form and evaluate polynomial regression models.

\hypertarget{sec:biglm}{%
    \subsection{biglm}\label{sec:biglm}}

biglm is described succinctly as

\begin{displayquote}
    bounded memory linear and generalized linear
    models\cite{lumley2013biglm}.
\end{displayquote}

biglm has been extended by other packages, and can integrate with
bigmemory matrices through biganalytics. The package is developed by
Dr.Thomas Lumley of the University of Auckland.