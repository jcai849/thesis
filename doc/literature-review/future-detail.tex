\hypertarget{sec:overview}{%
    \subsection{Overview}\label{sec:overview}}

future is introduced with the following summary:

\begin{displayquote}
    The purpose of this package is to provide a lightweight and unified
    future API for sequential and parallel processing of R expression via
    futures.
\end{displayquote}

The simplest way to evaluate an expression in parallel is to use
\mintinline{r}{x %<-% { expression }}
with \mintinline{r}{plan(multiprocess)}. This package
implements sequential, multicore, multisession, and cluster futures.
With these, R expressions can be evaluated on the local machine, in
parallel a set of local machines, or distributed on a mix of local and
remote machines. Extensions to this package implement additional
backends for processing futures via compute cluster schedulers etc.
Because of its unified API, there is no need to modify any code in order
switch from sequential on the local machine to, say, distributed
processing on a remote compute cluster. Another strength of this package
is that global variables and functions are automatically identified and
exported as needed, making it straightforward to tweak existing code to
make use of futures.\cite{bengtsson20} futures are abstractions for
values that may be available at some point in the future, taking the
form of objects possessing state, being either resolved and therefore
available immediately, or unresolved, wherein the process blocks until
resolution.

futures find their greatest use when run asynchronously. The future
package has the inbuilt capacity to resolve futures asynchronously,
including in parallel and through a cluster, making use of the parallel
package. This typically runs a separate process for each future,
resolving separately to the current R session and modifying the object
state and value according to it's resolution status.

\hypertarget{sec:comparison-with-non}{%
    \subsection{Comparison with Substitution and
        Quoting}\label{sec:comparison-with-non}}

R lays open a powerful set of metaprogramming functions, which bear
similarity to future. R expressions can be captured in a
\mintinline{r}{quote()}, then evaluated in an
environment with \mintinline{r}{eval()} at some point
in the future. Additionally,
\mintinline{r}{substitute()} substitutes any variables
in the expression passed to it with the values bound in an environment
argument, thus allowing ``non-standard evaluation'' in functions.

future offers a delay of evaluation as well, however such a delay is not
due to manual control of the programmer through
\mintinline{r}{eval()} functions and the like, but due
to background computation of an expression instead.

\hypertarget{sec:examples}{%
    \subsection{Example Usage}\label{sec:examples}}

Through substitution and quoting, R can, for example, run a console
within the language. Futures allows the extension of this to a parallel
evaluation scheme. Listing \cref{lst:console} gives a simple
implementation of this idea: a console that accepts basic expressions,
evaluating them in the background and presenting them upon request when
complete. Error handling and shared variables are not implemented.

\begin{listing}
    \begin{minted}{r}
library(future)

multicore.console <- function(){
    get.input <- function(){
        cat("Type \"e\" to enter an expression for",
            "evaluation \nand \"r\" to see",
            "resolved expressions\n", sep="")
        readline()
    }

    send.expr <- function(){
        cat("Multicore Console> ")
        input <- readline()
        futs[[i]] <<- future(eval(str2expression(input)))
        cat("\nResolving as: ", as.character(i), "\n")
    }

    see.resolved <- function(){
        for (i in 1:length(futs)){
            if (is(futs[[i]], "Future") &
                resolved(futs[[i]])) {
                cat("Resolved: ", as.character(i), " ")
                print(value(futs[[i]]))
            }
        }
    }

    plan(multicore)
    futs <- list()
    i <- 1
    while(TRUE){
        input <- get.input()
        if (input == "e") {
            send.expr()
            i <- i + 1
        } else if (input == "r") {
            see.resolved()
        } else {
            cat("Try again")
        }
    }
}

multicore.console()
\end{minted}
    \caption{Usage of future to implement a basic multicore console}
    \label{lst:console}
\end{listing}

\hypertarget{sec:extension-packages}{%
    \subsection{Extension Packages}\label{sec:extension-packages}}

\begin{description}
    \item[doFuture]
        \cite{bengtsson20do} provides an adapter for foreach\cite{microsoft20}
        that works on a future-based backend. Note that this does does not
        return \mintinline{r}{foreach()} calls as futures. The multicore features enabled with
        future are redundant over the existing parallel package, but because
        future backends can include other clusters, such as those provided by
        batchtools, there is some additional functionality, including additional
        degrees of control over backends.
    \item[future.batchtools]
        \cite{bengtsson19batch} provides a future API for
        batchtools\cite{lang17}, or equivalently, a batchtools backend for
        future. This allows the use of various cluster schedulers such as
        TORQUE, Slurm, Docker Swarm, as well as custom cluster functions.
    \item[future.apply]
        \cite{bengtsson20apply} provides equivalent functions to R's
        \mintinline{r}{apply} procedures, with a future backend enabling parallel,
        cluster, and other functionality as enabled by backends such as
        batchtools through future.batchtools.
    \item[future.callr]
        \cite{bengtsson19callr} provides a callr\cite{csardi20} backend to
        future, with all of the associated advantages and overhead. Callr
        ``calls R from R''. It provides functions to run expressions in a
        background R process, beginning a new session. An advantage of callr is
        that it allows more than 125 connections, due to not making use of
        R-specific connections. Additionally, no ports are made use of, unlike
        the SOCKcluster provided by the snow component of parallel.
    \item[furrr]
        \cite{vaughan18} allows the use of future as a backend to purrr
        functions. purrr is a set of functional programming tools for R,
        including map, typed map, reduce, predicates, and monads. Much of it is
        redundant to what already exists in R, but it has the advantage and goal
        of adhering to a consistent standard.
\end{description}

\hypertarget{sec:further-considerations}{%
    \subsection{Further Considerations}\label{sec:further-considerations}}

One initial drawback to future is the lack of callback functionality,
which would open enormous potential. However, this feature is made
available in the \emph{promises} package, which has been developed by
Joe Cheng at RStudio, which allows for user-defined handlers to be
applied to futures upon resolution\cite{Cheng19}.

Issues that aren't resolved by other packages include the copying of
objects referenced by future, with mutable objects thereby unable to be
directly updated by future (though this may be ameliorated with
well-defined callbacks). This also means that data movement is
mandatory, and costly; future raises an error if the data to be
processed is over 500Mb, though this can be overridden.

Referencing variables automatically is a major unsung feature of future,
though it doesn't always work reliably; future relies on code
inspection, and allows a \mintinline{r}{global} parameter to have manual
variable specification.

It seems likely that the future package will have some value to it's
use, especially if asynchronous processing is required on the R end; it
is the simplest means of enabling asynchrony in R without having to
manipulate networks or threads.
