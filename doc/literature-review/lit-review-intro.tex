% aim of lit review
In this chapter a broad overview of the existing literature is laid out and described, with several detailed case studies exploring key work in detail.
All such description is given with continual reference to the problem of expressing statistical algorithms over big data, in the context of an intended \lsr{} solution.
The general structure of this chapter is to explore strategies for handling large-scale data, then go into further detail in evaluating the existing solutions in the intended solution-space of \R{} packages.
This forms something of an understanding of what strategies and interfaces exist.
This is mirrored with consideration of implementation to close the literature review; aspects of the \R{} language that will have an effect on the solution, as well as a discussion of well-known algorithms for larger-than-memory data, with a particular focus on distributed hash tables.

In detail, the strategies for handling large scale data are covered in \cref{sec:lit-review-strategies}, with the main pattern emerging of the value in distributing data.
\Cref{sec:large-scale-features,sec:survey-dist-comp-sys} possess the broad context of large-scale and distributed computing systems that may be used for statistical computing, beyond the sphere of \R{}, and provide something of a survey of such systems.
\Cref{sec:survey-r-local-packages,sec:survey-r-dist-packages,sec:survey-r-deriv,sec:survey-r-stat-model-packages} serve to narrow the scope to packages in \R{} for large data, assessing them from various angles.
\Cref{sec:review-iteration-sparklyr,sec:future-detail,sec:review-foreach,sec:disk-frame-study} serve as more detailed studies of the \R{} packages \pkg{sparklyr}, \pkg{future}, \pkg{foreach}, and \pkg{disk.frame} respectively.
From there, \cref{sec:concurrency-r,sec:eval-r} consider the aspects of \R{} that will have some effect on the solution, and the chapter rounds off with an illustration of how algorithms change upon being distributed, with an analysis of the core literature on distributed hash tables in \cref{sec:dht}.

In seeking to engage with the context of software systems for large data, it will be seen that many projects and systems aren't associated with standard means of reference; as such, this literature review involves an empirical aspect to it, in describing experimentation and basic usage of some of these systems.
Major developments in contemporary statistical computing are typically published alongside \R{} code implementation, usually in the form of an R package, which is a mechanism for extending \R{} and sharing functions.
As of October 2023, the Comprehensive \R{} Archive Network (CRAN) hosts just shy of 20,000 available packages\cite{team20:_r}.

The main patterns that can be drawn from this literature review is the lack of satisfactory solutions in solving the problem of expressing statistical algorithms over a larger-than-memory dataset in \R{}, and perhaps most importantly -- that it is indeed a hard problem.
