\hypertarget{sec:hadoop-1}{%
    \subsection{Hadoop}\label{sec:hadoop-1}}

Apache \pkg{Hadoop} is a collection of utilities that facilitates cluster
computing. Jobs can be sent for parallel processing on the cluster
directly to the utilities using .jar files, ``streamed'' using any
executable file, or accessed through language-specific APIs.

The project began in 2006, by Doug Cutting, a Yahoo employee, and Mike
Cafarella. The inspiration for the project was a paper from Google
describing the Google File System (described in
\cite{ghemawat2003google}, which was followed by another Google paper
detailing the MapReduce programming model, \cite{dean2004mapreduce}.

Hadoop consists of a memory part, known as \pkg{Hadoop} Distributed File
System (HDFS), described in subsection\cref{sec:hdfs},
and a processing part, known as MapReduce, described in
subsection\cref{sec:mapreduce}.

In operation, \pkg{Hadoop} splits files into blocks, then distributes them
across nodes in a cluster, where they are then processed by the node.
This creates the advantage of data locality, wherein data is processed
by the node they exist in.

Hadoop has seen extensive industrial use as the premier big data
platform upon it's release. In recent years it has been overshadowed by
Spark, due to the greater speed gains offered by \pkg{Spark}. The key reason
Spark is so much faster than \pkg{Hadoop} comes down to their different
processing approaches: \pkg{Hadoop} MapReduce requires reading from disk and
writing to it, for the purposes of fault-tolerance, while \pkg{Spark} can run
processing entirely in-memory. However, in-memory MapReduce is provided
by another Apache project, Ignite\cite{zheludkov2017high}.

\hypertarget{sec:hdfs}{%
    \subsubsection{Hadoop Distributed File System}\label{sec:hdfs}}

The file system has 5 primary services.

\begin{description}

    \item[Name Node]
        Contains all of the data and manages the system. The master node.
    \item[Secondary Name Node]
        Creates checkpoints of the metadata from the main name node, to
        potentially restart the single point of failure that is the name node.
        Not the same as a backup, as it only stores metadata.
    \item[Data Node]
        Contains the blocks of data. Sends ``Heartbeat Message'' to the name
        node every 3 seconds. If two minutes passes with no heartbeat message
        from a particular data node, the name node marks it as dead, and sends
        it's blocks to another data node.
    \item[Job Tracker]
        Receives requests for MapReduce from the client, then queries the name
        node for locations of the data.
    \item[Task Tracker]
        Takes tasks, code, and locations from the job tracker, then applies such
        code at the location. The slave node for the job tracker.
\end{description}

HDFS is written in Java and C. It is described in more detail in
\cite{shvachko2010hadoop}

\hypertarget{sec:mapreduce}{%
    \subsubsection{MapReduce}\label{sec:mapreduce}}

MapReduce is a programming model consisting of map and reduce staps,
alongside making use of keys.

\begin{description}

    \item[Map]
        applies a ``map'' function to a dataset, in the mathematical sense of
        the word. The output data is temporarily stored before being shuffled
        based on output key, and sent to the reduce step.
    \item[Reduce]
        produces a summary of the dataset yielded by the map operation
    \item[Keys]
        are associated with the data at both steps. Prior to the application of
        mapping, the data is sorted and distributed among data nodes by the
        data's associated keys, with each key being mapped as a logical unit.
        Likewise, mapping produces output keys for the mapped data, and the data
        is shuffled based upon these keys, before being reduced.
\end{description}

After sorting, mapping, shuffling, and reducing, the output is
collected, sorting by the second keys and given as final output.

The implementation of MapReduce is provided by the HDFS services of job
tracker and task tracker. The actual processing is performed by the task
trackers, with scheduling using the job tracker, but other scheduling
systems are available to be made use of.

Development at Google no longer makes as much use of MapReduce as they
originally did, using stream processing technologies such as MillWheel,
rather than the standard batch processing enabled by
MapReduce\cite{akidau2013millwheel}.

\hypertarget{sec:spark}{%
    \subsection{Spark}\label{sec:spark}}

\pkg{Spark} is a framework for cluster computing\cite{zaharia2010spark}. Much
of it's definition is in relation to \pkg{Hadoop}, which it intended to
improve upon in terms of speed and usability for certain tasks.

It's fundamental operating concept is the Resiliant Distributed Dataset
(RDD), which is immutable, and generated through external data, as well
as actions and transformations on prior RDD's. The RDD interface is
exposed through an API in various languages, including R.

Spark requires a distributed storage system, as well as a cluster
manager; both can be provided by \pkg{Hadoop}, among others.

Spark is known for possessing a fairly user-friendly API, intended to
improve upon the MapReduce interface. Another major selling point for
Spark is the libraries available that have pre-made functions for RDD's,
including many iterative algorithms. The availability of broadcast
variables and accumulators allow for custom iterative programming.

Spark has seen major use since it's introduction, with effectively all
major big data companies having some use of \pkg{Spark}. It's features and
implementations are outlined in \cite{zaharia2016apache}.

\hypertarget{sec:h2o}{%
    \subsection{H2O}\label{sec:h2o}}

The \pkg{H2O} software bills itself as,

\begin{displayquote}
    an in-memory platform for distributed, scalable machine learning. \pkg{H2O}
    uses familiar interfaces like R, Python, Scala, Java, JSON and the Flow
    notebook/web interface, and works seamlessly with big data technologies
    like \pkg{Hadoop} and \pkg{Spark}. \pkg{H2O} provides implementations of many popular
    algorithms such as GBM, Random Forest, Deep Neural Networks, Word2Vec
    and Stacked Ensembles. \pkg{H2O} is extensible so that developers can add data
    transformations and custom algorithms of their choice and access them
    through all of those clients.
\end{displayquote}

H2O typically runs on HDFS, along with \pkg{Spark} for computation and bespoke
data structures serving as the backbone of the architecture.

H2O can communicate with \R through a REST api. Users write functions in
R, passing user-made functions to be applied on the objects existing in
the \pkg{H2O} system\cite{h2o.ai:_h2o}.

The company \pkg{H2O} is backed by \$146M in funding, partnering with large
institutions in the financial and tech world. Their business model
follows an open source offering with the same moniker as the company,
and a small set of heavily-marketed proprietary software in aid of it.
They have some important figures working with them, such as Matt Dowle,
creator of \pkg{data.table}.
