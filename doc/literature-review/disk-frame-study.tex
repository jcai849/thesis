\subsection{Introduction}\label{sec:disk-frame-introduction}

\pkg{disk.frame} works through two main principles: chunking, and generic
function implementation (alongside special functions). Another component
that isn't mentioned in the explanation, but is crucial to performance,
is the parallelisation offered transparently by the package.

\pkg{disk.frame} is developed by Dai ZJ.

\subsection{Chunks and Chunking}\label{sec:chunking}

\subsubsection{Chunk Representation}\label{sec:chunk-representation}

disk.frames are actually references to numbered \file{fst} files in a
folder, with each file serving as a chunk. This is made use of through
manipulation of each chunk separately, sparing RAM from dealing with a
single monolithic file\cite{zj19:_inges_data}.

\pkg{fst} is a means of serialising dataframes, as an alternative to \file{RDS}
files\cite{klik19}. It makes use of an extremely fast compression
algorithm developed at facebook, with the \R{} package enabling fst written on in \cref{sec:fst}.

From inspection of the source code, \pkg{data.table} manipulations are enabled
directly through transformations of each chunk to data.tables through
the \pkg{fst} backend.

\subsubsection{Chunk Usage}\label{sec:making-chunks}

Chunks are created transparently by \pkg{disk.frame}; the user can
theoretically remain ignorant of chunking. In \R{}, the disk.frame object
serves as a reference to the chunk files. Operations on disk.frame
objects are by default lazy, waiting until the
\code{collect} command to perform the
collected operations and pull the chunks into \R{} as a single data.table.
As noted in \cite{zj19:_simpl_verbs_lazy_evaluat}, this form of lazy
evaluation is similar to the implementation of \pkg{sparklyr}.

Chunks are by default assigned rows through hashing the source rows, but
can be composed of individual levels of some source column, which can
provide an enormous efficiency boost for grouped operations, where the
computation visits the data, rather than the other way around.

Chunks can be manipulated individually, having individual ID's, through
\code{get-chunk}, as well as added or
removed from additional \file{fst} files directly, through
\code{add-chunk} and
\code{remove-chunk}, respectively.

In a computationally intensive procedure, the rows can be rearranged
between chunks based on a particular column level as a hash, through
functions such as \code{rechunk}.

\subsection{Functions}\label{sec:functions}

The disk.frame object has standard procedures for construction and
access. disk.frame can be constructed from data.frames and data.tables
through \code{as-disk-frame}, single or
multiple \file{csv} files through
\code{csv-to-disk-frame}, as well as \file{zip}
files holding \file{csv} files. Time can be saved later on through the
application of functions to the data during the conversion, as well as
specifying what to chunk by, keeping like data together. The process of
breaking up data into chunks is referred to by \pkg{disk.frame} as
``sharding'', enabled for data.frames and data.tables through the
\code{shard} function.

After creating a disk.frame object, functions can be applied directly to
all chunks invisibly through using the
\code{cmap} family of functions in a form
similar to base \R{} \code{apply}

A highly publicised aspect of \pkg{disk.frame} is the functional
cross-compatibility with \pkg{dplyr} verbs. These operate on disk.frame
objects lazily, and are applied through translation by \pkg{disk.frame}; they
are just S3 methods defined for the disk.frame class. They are fully
functioning, with the exception of \code{group-by} (and its
data.table cognate, \code{data-table-by}, considered in more detail in
\cref{sec:spec-cons-group-by}).

Beyond higher-order functions and \pkg{dplyr} or \pkg{data.table} analogues for data
manipulation, the direct modelling function of
\code{dfglm} is implemented to allow for
fitting glms to the data. From inspection of the source code, the
function is a utility wrapper for streaming \pkg{disk.frame} data by default
into \pkg{bigglm}, a \pkg{biglm} derivative.

\subsubsection{Grouping}\label{sec:spec-cons-group-by}

For a select set of functions, \pkg{disk.frame} offers a transparent grouped
\code{summarise}. These are mainly composed
of simple statistics such as \code{mean},
\code{min}, etc.

For other grouped functions, there is more complexity involved, due to
the chunked nature of \pkg{disk.frame}. When functions are applied, they are
by default applied to each chunk. If groups don't correspond injectively
to chunks, then the syntactic chunk-wise summaries and their derivatives
may not correspond to the semantic group-wise summaries expected. For
example, summarising the median is performed by using a
median-of-medians method; finding the overall median of all chunks'
respective medians. Therefore, computing grouped medians in \pkg{disk.frame}
result in estimates only -- this is also true of other software, such
as \pkg{Spark}, as noted in \cite{zj19:_group_by}.

Grouped functions are thereby divided into one-stage and two-stage;
one-stage functions ``just work'' with the
\code{group-by} function, and two-stage
functions requiring manual chunk aggregation (using
\code{chunk-group-by} and \code{chunk-summarize}), followed by an
overall collected aggregation (using regular
\code{group-by} and
\code{summarise}). \Textcite{zj19:_group_by}
points out that explicit two-stage approach is similar to a MapReduce
operation.

Custom one-stage functions can be created, where user-defined chunk
aggregation and collected aggregation functions are converted into
one-stage functions by
disk.frame\cite{zj19:_custom_one_stage_group_by_funct}. These take
the forms
\code{chunk-agg} and
\code{collected-agg}
respectively, where ``fn'' is used as the name of the function,
and appended to the defined name by \pkg{disk.frame}, through
meta-programming.

To de-complicate the situation, but add one-off computational overhead,
chunks can be rearranged to correspond to groups, thereby allowing for
one-stage summaries just through
\code{chunk-summarize}, and exact
computations of group medians.

\subsection{Parallelism}\label{sec:parallelisation}

An essential component of \pkg{disk.frame}'s speed is parallelisation; as
chunks are conceptually separate entities, function application to each
can take place with no side effects to other chunks, and can therefore
be trivially parallelised.

For parallelisation, future is used as the backend package, with most
function mappings on chunks making use of \code{future-lapply}
to have each chunk mapped with the intended function in parallel. Future
is a package with complexities in its own right; I have written more on
future in \cref{sec:future-detail}

future is initialised with access to cores through the wrapper function,
\code{setup-disk-frame}\cite{zj19:_key}.
This sets up the correct number of workers, with the minimum of workers
and chunks being processed in parallel.

An important aspect to parallelisation through future is that, for
purposes of cross-platform compatibility, new \R{} processes are started
for each worker\cite{zj19:_using}. Each process will possess its own
environment, and \pkg{disk.frame} makes use of future's detection capabilities
to capture external variables referred to in calls, and send them to
each worker.

\subsection{Relevance}\label{sec:relevance}

\pkg{disk.frame} serves as an example of a very well-received and used package
for larger-than-RAM processing. The major keys to its success have been
its chart-topping performance, even in comparison with \pkg{dask} and \proglang{Julia},
and its user-friendliness enabled through procedural transparency and
well-articulated concepts.

\pkg{disk.frame} as a concept also lends itself well to extension:

The storage of chunks is currently file-based and managed by an
operating system; if fault tolerance was desired, HDFS support for chunk
storage would likely serve this purpose well.

Alternatively, OS-based file manipulation could be embraced in greater
depth, focussing on interaction with faster external tools for file
manipulation; this would lead to issues with portability, but with
reasonable checks, could enable great speedups through making use of
system utilities such as \code[sh]{sort} on UNIX-based systems.

The type of file may also be open to extension, with other file formats
competing for high speeds and cross-language communication including \pkg{feather}, developed by Wes~McKinney and Hadley~Wickham\cite{wes16}.

In terms of finer-grained extension, more functionality for direct
manipulation of individual chunks would potentially aid computation when
performing iterative algorithms and others of greater complexity.
