\hypertarget{sec:introduction}{%
	\subsection{Introduction}\label{sec:foreach-introduction}}

foreach introduces itself on CRAN with the following description:

\begin{displayquote}
	Support for the foreach looping construct. Foreach is an idiom that
	allows for iterating over elements in a collection, without the use of
	an explicit loop counter. This package in particular is intended to be
	used for its return value, rather than for its side effects. In that
	sense, it is similar to the standard lapply function, but doesn't
	require the evaluation of a function. Using foreach without side effects
	also facilitates executing the loop in parallel.
\end{displayquote}

From the user end, the package is conceptually simple, revolving
entirely around a looping construct and the one-off backend
registration.

The principal goal of the package, which it hasn't strayed from, is the
enabling of parallelisation through backend transparency within the
foreach construct. Notably, more complex functionality, such as side
effects and parallel recurrance, are not part of the package's
intention.

Thus, the primary driver for the practicality of the package, beyond the
support offered for parallel backends, is the backends themselves,
currently enabling a broad variety of parallel systems.

foreach is developed by Steve Weston and Hoong Ooi.

\hypertarget{sec:usage}{%
	\subsection{Usage}\label{sec:usage}}

foreach doesn't require setup for simple serial execution, but parallel
backends require registration by the user, typically with a single
function as in the registration for doParallel,
\mintinline{r}{registerDoParallel()}.

The syntax of foreach consists of a
\mintinline{r}{foreach()} function call next to a
\mintinline{r} operator, and some expression to the
right\cite{weston19:_using}. Without loss in generality, the syntactic
form is given in \ref{lst:syntax}.

\begin{listing}
	\begin{minted}{r}
foreach(i=1:n) %do% {expr}
\end{minted}
	\caption{Standard foreach syntax}
	\label{lst:syntax}
\end{listing}

The \mintinline{r}{foreach()} function can take other
arguments including changing the means of combination along iterations,
whether iterations should be performed in order, as well as the export
of environmental variables and packages to each iteration instance.

In addition to \mintinline{r}, other binary operators can be appended
or substituted. Parallel iteration is performed by simply replacing
\mintinline{r} with \mintinline{r}. Nested loops can be created by
inserting \mintinline{r} between main and nested foreach functions,
prior to the \mintinline{r} call\cite{weston19:_nestin_loops}. The
last step to composition of foreach as capable of list comprehension is
the filtering function \mintinline{r}, which filters iterables based
on some predicate to control evaluation.

\hypertarget{sec:implementation}{%
	\subsection{Implementation}\label{sec:implementation}}

The mechanism of action in foreach is often forgotten in the face of the
atypical form of the standard syntax. Going one-by-one, the
\mintinline{r}{foreach()} function returns an iterable
object, \mintinline{r} and derivatives are binary functions operating
on the iterable object returned by
\mintinline{r}{foreach()} on the left, and the
expression on the right; the rightmost expression is simply captured as
such in \mintinline{r}. Thus, the main beast of burder is the
\mintinline{r} function, where the evaluation of the iteration takes
place.

In greater detail, \mintinline{r} captures and creates environments,
enabling sequential evaluation. \mintinline{r} captures the
environment of an expression, as well taking as a formal parameter a
vector of names of libraries used in the expression, then passing that
to the backend, which will in turn do additional work on capturing
references to variables in expressions and adding them to evaluation
environment, as well as ensure packages are loaded on worker nodes.

\mintinline{r} and \mintinline{r}, after correct error checking,
send calls to \mintinline{r}{getDoSeq()} and
\mintinline{r}{getDoPar()} respectively, which return
lists determined by the registered backend, which contain a function
used backend, used to operate on the main expression along with other
environmental data.

foreach depends strongly upon the iterators package, which gives the
ability to construct custom iterators. These custom iterators can be
used in turn with the \mintinline{r}{foreach()}
function, as the interface to them is transparent.

\hypertarget{sec:form-iter}{%
	\subsection{Form of Iteration}\label{sec:form-iter}}

The name of the package and function interface refer to the
\mintinline{r}{foreach} programming language construct, present in many other
languages. By definition, the \mintinline{r}{foreach} construct performs
traversal over some collection, not necessarily requiring any traversal
order. In this case, the collection is an iterator object or an object
coercible to one, but in other languages with foreach as part of the
core language, such as python (whose for loop is actually only a foreach
loop), collections can include sets, lists, and a variety of other
classes which have an \mintinline{r}{__iter__} and
\mintinline{r}{__next__} defined\cite{python2020iter}.

Due to the constraints imposed by a foreach construct, loop optimisation
is simplified relative to a for loop, and the lack of explicit traversal
ordering permits parallelisation, which is the primary reason for usage
of the \mintinline{r}{foreach} package. The constraints are not insignificant
however, and they do impose a limit on what can be expressed through
their usage. Most notably, iterated functions, wherein the function
depends on it's prior output, are not necessarily supported, and
certainly not supported in parallel. This is a result of the order of
traversal being undefined, and when order is essential to maintain
coherent state, as in iterated functions, the two concepts are mutually
exclusive.

In spite of the constraints, iterated functions can actually be emulated
in foreach through the use of destructive reassignment within the passed
expression, or through the use of stateful iterators. Examples of both
are given in \ref{lst:serial}\ref{lst:serial-iter}.

\begin{listing}
	\begin{minted}{r}
x <- 10
foreach(i=1:5) %do% {x <- x+1}
\end{minted}
	\caption{Serial iterated function through destructive reassignment}
	\label{lst:serial}
\end{listing}

\begin{listing}
	\begin{minted}{r}
addsone <- function(start, to) {
	nextEl <- function(){
		start <<- start + 1
		if (start >= to) {
			stop('StopIteration')
		}
		start}
	obj <- list(nextElem=nextEl)
	class(obj) <- c('addsone', 'abstractiter', 'iter')
	obj
}

it <- addsone(10, 15)
nextElem(it)

foreach(i = addsone(10, 15), .combine = c) %do% i
\end{minted}
	\caption{Serial iterated function through creation of a stateful iterator}
	\label{lst:serial-iter}
\end{listing}

As alluded to earlier, the functionality breaks down when attempting to
run them in parallel. \ref{lst:parallel}\ref{lst:parallel-iter}
demonstrate attempts to evaluate these iterated functions in parallel.
They only return a list of 5 repetitions of the same ``next'' number,
not iterating beyond it.

\begin{listing}
	\begin{minted}{R}
cl <- makeCluster(2)
doParallel::registerDoParallel(cl)
x <- 10
foreach(i=1:5) %dopar% {x <- x+1}
\end{minted}
	\caption{Parallel Iteration attempt through destructive reassignment}
	\label{lst:parallel}
\end{listing}

\begin{listing}
	\begin{minted}{R}
doParallel::registerDoParallel
foreach(i = addsone(10, 15), .combine = c) %dopar% i
\end{minted}
	\caption{Parallel Iteration attempt through a stateful iterator}
	\label{lst:parallel-iter}
\end{listing}

\subsection*{Extensions}

The key point of success in foreach is it's backend extensibility,
without which, foreach would lack any major advantages over a standard
\mintinline{r}{for} loop.

Other parallel backends are enabled through specific functions made
available by the foreach package. The packages define their parallel
evaluation procedures with reference to the iterator and accumulator
methods from foreach.

Numerous backends exist, most notably:

\begin{description}
	\item[doParallel]
		the primary parallel backend for foreach, using the parallel
		package\cite{corporation19}.
	\item[doRedis]
		provides a Redis backend, through the redux package\cite{lewis20}.
	\item[doFuture]
		uses the future package to make use of future's many
		backends\cite{bengtsson20do}.
	\item[doAzureParallel]
		allows for direct submission of parallel workloads to an Azure Virtual
		Machine\cite{hoang20}.
	\item[doMPI]
		provides MPI access as a backend, using Rmpi\cite{weston17}.
	\item[doRNG]
		provides for reproducible random number usage within parallel
		iterations, using L'Ecuyer's method; provides
		\mintinline{r}\cite{gaujoux20}.
	\item[doSNOW]
		provides an ad-hoc cluster backend, using the snow
		package\cite{dosnow19}.
\end{description}

\hypertarget{relevance}{%
	\subsection{Relevance}\label{relevance}}

foreach serves as an example of a well-constructed package supported by
it's transparency and extensibility.

For packages looking to provide any parallel capabilities, a foreach
extension would certainly aid it's potential usefulness and visibility.
