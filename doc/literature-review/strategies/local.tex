
While not specifically engaging with larger-than-memory data, a number
of packages take advantage of various parallel strategies in order to
process large datasets efficiently. \pkg{multicore} is one such
package, now subsumed into the \pkg{parallel} package, that grants
functions that can make direct use of multiprocessor systems, thereby
reducing the processing time in proportionality to the number of
processors available on the system.

\pkg{data.table} also makes use of multi-processor systems, with many
operations involving threading in order to rapidly perform operations on
its dataframe equivalent, the \pkg{data.table}.

In spite of all of these potential solutions, a major constraint remains
in that only a single machine is used. As long as there is only one
machine available, bottlenecks form and no redundancy protection is
offered in real-time in the event of a crash or power outage.

The first steps typically taken to manage larger-than-memory data is to
shift part of the data into secondary storage, which generally possesses
significantly more space than main memory.

This is the approach taken by the \pkg{disk.frame} package, developed
by Dai ZJ. \pkg{disk.frame} provides an eponymously named dataframe
replacement class, which is able to represent a dataset far larger than
RAM, constrained now only by disk size\cite{zj20}.

The mechanism of \pkg{disk.frame} is introduced on its homepage with the
following explanation:

\cqu{zj20}

It works through two main principles: chunking, and an array of methods
taking advantage of data.frame generics, including \pkg{dplyr} and
\pkg{data.table} functions.

Another component that isn't mentioned in the explanation, but is
crucial to performance, is the parallelisation offered transparently by
the package.

disk.frames are actually references to numbered \file{fst} files in a
folder, with each file serving as a chunk.
This is made use of through manipulation of each chunk separately,
sparing RAM from dealing with a single monolithic
file\cite{zj19:_inges_data}.
Fst is a means of serialising dataframes, as an alternative to RDS
files\cite{klik19}.
It makes use of an extremely fast compression algorithm developed at
facebook.
Functions are usually mapped over chunks using some functional, but more
complex functions such as those implementing a glm require custom
solutions; as an example the direct modelling function of
\code{dfglm} is implemented to allow
for fitting glms to the data.
From inspection of the source code, the function is a utility wrapper
for streaming \pkg{disk.frame} data by default into \pkg{bigglm}, a \pkg{biglm}
derivative.
For grouped or aggregated functions, there is more complexity involved,
due to the chunked nature of \pkg{disk.frame}.
When functions are applied, they are by default applied to each chunk.
If groups don't correspond injectively to chunks, then the syntactic
chunk-wise summaries and their derivatives may not correspond to the
semantic group-wise summaries expected.
For example, summarising the median is performed by using a
median-of-medians method; finding the overall median of all chunks'
respective medians.

Therefore, computing grouped medians in \pkg{disk.frame} result in estimates
only --- this is also true of other software, such as \pkg{spark}, as noted in
\cite{zj19:_group_by}.
For parallelisation, \pkg{future} is used as the backend package, with most
function mappings on chunks making use of
\code{future-lapply}
to have each chunk mapped with the intended function in parallel.
future is initialised with access to cores through the wrapper function,
\code{setup-disk-frame}\cite{zj19:_key}.
This sets up the correct number of workers, with the minimum of workers
and chunks being processed in parallel.
An important aspect to parallelisation through \pkg{future} is that, for
purposes of cross-platform compatibility, new \R{} processes are started
for each worker\cite{zj19:_using}.
Each process will possess its own environment, and \pkg{disk.frame} makes use
of future's detection capabilities to capture external variables
referred to in calls, and send them to each worker.
The strategy taken by \pkg{disk.frame} has several inherent
limitations, however. \pkg{disk.frame} allows only embarrassingly
parallel operations for custom operations as part of a
split-apply-combine (MapReduce) pattern.
While there may theoretically be future provision for non-embarrassingly
parallel operations, a significant limitation to real-time operation is
the massive slowdown brought by the data movement from disk to RAM and
back.

