
In the search for a distributed system for statistics, the world outside
of \R is not remotely barren. The central issue with non-\R distributed
systems is that their focus is very obviously not statistics, and this
shows in the level of support the platforms provide for statistical
purposes.

The classical distributed system for high-performance computing is \pkg{MPI}.
\R actually has a high-level interface to \pkg{MPI} through the \pkg{rmpi}
package. This package is excellent, but extremely low-level, offering
little more than wrappers around \pkg{MPI} functions. For the statistician who
just wants to implement a model for a large dataset, such concern with
minutiae is prohibitive.

\pkg{Hadoop} and \pkg{Spark} are two closely related systems which were mentioned
earlier.

\pkg{Apache Hadoop} is a collection of utilities that facilitates cluster
computing.

Jobs can be sent for parallel processing on the cluster directly to the
utilities using \file{.jar} files, ``streamed'' using any executable file, or
accessed through language-specific APIs.

The project began in 2006, by Doug Cutting, a Yahoo employee, and Mike
Cafarella.

The inspiration for the project was a paper from Google describing the
Google File System (described in \textcite{ghemawat2003google}), which was
followed by another Google paper detailing the MapReduce programming
model, \textcite{dean2004mapreduce}.

\pkg{Hadoop} consists of a file-store component, known as Hadoop Distributed
File System (HDFS), and a processing component, known as MapReduce.

In operation, \pkg{Hadoop} splits files into blocks, then distributes them
across nodes in a cluster (HDFS), where they are then processed by the
node in parallel (MapReduce). This creates the advantage of data
locality, wherein data is processed by the node they exist in.

\pkg{Hadoop} has seen extensive industrial use as the premier big data
platform upon its release.

In recent years it has been overshadowed by \pkg{Spark}, due to the greater
speed gains offered by \pkg{Spark} for many problem sets.

\pkg{Spark} was developed with the shortcomings of \pkg{Hadoop} in mind; Much of
its definition is in relation to \pkg{Hadoop}, which it intended to improve
upon in terms of speed and usability for certain
tasks\cite{zaharia2010spark}.

Its fundamental operating concept is the Resiliant Distributed Dataset
(RDD), which is immutable, and generated through external data, as well
as actions and transformations on prior RDD's.

The RDD interface is exposed through an API in various languages,
including \R, however it appears to be abandoned to some degree, having
removed from the CRAN repository at 2020-07-10 due to failing checks.

\pkg{Spark} requires a distributed storage system, as well as a cluster
manager; both can be provided by \pkg{Hadoop}, among others.

\pkg{Spark} is known for possessing a fairly user-friendly API, intended to
improve upon the MapReduce interface.

Another major selling point for \pkg{Spark} is the libraries available that
have pre-made functions for RDD's, including many iterative algorithms.

The availability of broadcast variables and accumulators allow for
custom iterative programming.

\pkg{Spark} has seen major use since its introduction, with effectively all
major big data companies having some use of Spark.

In the \proglang{Python} world, the closest match to a high-level distributed
system that could have statistical application is given by the python
library \pkg{dask}\cite{rocklin2015dask}. \pkg{dask} offers
dynamic task scheduling through a central task graph, as well as a set
of classes that encapsulate standard data manipulation structures such
as \pkg{NumPy} arrays and \pkg{Pandas} dataframes.

The main difference is that the \pkg{dask} classes take advantage of
the task scheduling, including online persistence across multiple nodes.
\pkg{dask} is a large and mature library, catering to many use-cases,
and exists largely in the \proglang{Python} ``Machine Learning'' culture in
comparison to the \R ``Statistics'' culture. Accordingly, the focus is
more tuned to the \proglang{Python} software developer putting existing ML models
into a large-scale capacity. Of all the distributed systems assessed so
far, \pkg{dask} comes the closest to what an ideal platform would
look like for a statistician, but it misses out on the statistical
ecosystem of \R, provides only a few select classes, and is tied entirely
to the structure of the task graph.
