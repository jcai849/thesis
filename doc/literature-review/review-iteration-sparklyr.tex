\subsection{Introduction}\label{subsec:sparklyr-introduction}

Given that iteration is cited by a principal author of \pkg{Spark} as a
motivating factor in it's development when compared to \pkg{Hadoop}, it is
reasonable to consider whether the most popular \R{} interface to \pkg{Spark},
\pkg{sparklyr}, has support for iteration\cites{zaharia2010spark,luraschi20}.
One immediate hesitation to the suitability of \pkg{sparklyr} to iteration is
the syntactic rooting in \pkg{dplyr}; \pkg{dplyr} is a ``Grammar of Data
Manipulation'' and part of the tidyverse, which in turn is an ecosystem
of packages with a shared philosophy\cite{wickham2019welcome}\cite{wickham2016r}.

The promoted paradigm is functional in nature, with iteration using for
loops in \R{} being described as ``not as important'' as in other
languages; map functions from the tidyverse purrr package are instead
promoted as providing greater abstraction and taking much less time to
solve iteration problems. Maps do provide a simple abstraction for
function application over elements in a collection, similar to internal
iterators, however they offer no control over the form of traversal, and
most importantly, lack mutable state between iterations that standard
loops or generators allow\cite{cousineau1998functional}. A common
functional strategy for handling a changing state is to make use of
recursion, with tail-recursive functions specifically referred to as a
form of iteration in \cite{abelson1996structure}. Reliance on recursion
for iteration is naively non-optimal in \R{} however, as it lacks tail-call
elimination and call stack optimisations\cite{rcore2020lang}; at present
the elements for efficient, idiomatic functional iteration are not
present in R, given that it is not as functional a language as the
tidyverse philosophy considers it to be, and \pkg{sparklyr}'s attachment to
the the ecosystem prevents a cohesive model of iteration until said
elements are in place.

\subsection{Iteration}\label{iteration}

Iteration takes place in \pkg{Spark} through caching results in memory,
allowing faster access speed and decreased data movement than
MapReduce\cite{zaharia2010spark}. \pkg{sparklyr} can use this functionality
through the \code{tbl-cache} function to
cache \pkg{Spark} dataframes in memory, as well as caching upon import with
\code{memory-true} as a formal parameter to \code{sdf-copy-to}.

Iteration can also make use of persisting \pkg{Spark} Dataframes to memory,
forcing evaluation then caching; performed in \pkg{sparklyr} through
\code{sdf-persist}.

The Babylonian method for calculating a square root is a simple
iterative procedure, used here as an example. A standard form in \R{} with
non-optimised initial value is given in \cref{lst:basicbab}.

\src{basicbab}{Simple Iteration with the Babylonian Method}


This iterative function is trivial, but translation to \pkg{sparklyr} is not
entirely so.

The first aspect that must be considered is that \pkg{sparklyr} works on \pkg{Spark}
Data Frames; the variables x and S must be copied to \pkg{Spark} with the
aforementioned \code{sdf-copy-to}
function.

The execution of the function in \pkg{Spark} is the next consideration, and
sparklyr provides two means for this to occur;
\code{spark-apply} evaluates arbitrary R
code over an entire data frame. The means of operation vary across \pkg{Spark}
versions, ranging from launching and running RScripts in \pkg{Spark} 1.5.2, to
Apache Arrow conversion in \pkg{Spark} 3.0.0.

The evaluation strategy of 1.5.2 is unsuitable in this instance as it is
excessive overhead to launch RScripts every iteration.

The other form of evaluation is through using \pkg{dplyr} generics, which is
what will be made use of in this example.

An important aspect of consideration is that \pkg{sparklyr} methods for \pkg{dplyr}
generics execute through a translation of the formal parameters to \pkg{Spark}
SQL. This is particularly relevant in that separate \pkg{Spark} Data Frames
can't be accessed together as in a multi-variable function. In addition,
very R-specific functions such as those from the \code{stats} and
\code{lib-matrix} core libraries are not able to be evaluated, as there is
no \proglang{Spark SQL} cognate for them. The SQL query generated by the methods
can be accessed and ``explained'' through
\code{show-query} and
\code{explain} respectively; When attempting
to combine two \pkg{Spark} Data Frames in a single query without joining them,
\code{show-query} reveals that the Data
Frame that is referenced through the .data variable is
translated, but the other Data Frame has it's list representation passed
through, which \proglang{Spark SQL} doesn't have the capacity to parse; an example
is given in \cref{lst:computer-no} (generated through listing
\cref{lst:bad}), showing an attempt to create a new column from the
difference between two separate Data Frames

\src{bad}{Attempt in \R{} to form new column from the difference between two separate \pkg{Spark} data frames S and x}

\src{computer-no}{\proglang{Spark SQL} query generated from attempt to form the difference from two separate data frames}

Global variables that evaluate to \proglang{SQL}-friendly objects can be passed and
are evaluated prior to translation. An example is given through
\cref{lst:global-ok}, generated through \cref{lst:ok-generator}, where
the difference between a variable holding a numeric and a \pkg{Spark} Data
Frame is translated into the evaluation of the variable, transformed to
a float for \proglang{Spark SQL}, and its difference with the \pkg{Spark} Data Frame, referenced directly.

\src{global-ok}{Spark SQL query generated from attempt to form the difference between a data frame and a numeric}

\src{ok-generator}{Capacity in \pkg{sparklyr} to form new column from the difference between a \pkg{Spark} data frame and a numeric}

A reasonable approach to implementing a Babylonian method in \pkg{sparklyr} is
then to combine S and x in one dataframe, and iterate
within columns.

\src{sparklyr-bab}{Babylonian method implementation using \pkg{sparklyr}}

\subsection{Conclusion}\label{subsec:sparklyr-conclusion}

\pkg{sparklyr} is excellent when used for what it is designed for. Iteration,
in the form of an iterated function, does not appear to be part of this
design; this was clear in the abuse required to implement a simple
iterated function in the form of the Babylonian Method. Furthermore, all
references to ``iteration'' in the primary \pkg{sparklyr} literature refer
either to the iteration inherent in the inbuilt \pkg{Spark} ML functions, or
the ``wrangle-visualise-model'' process popularised by Hadley
Wickham\cite{luraschi2019mastering}\cite{wickham2016r}. None of such
references connect with iterated functions.

Thus, it is fair to conclude that \pkg{sparklyr} is incapable of sensible
iteration of arbitrary \R{} code beyond what maps directly to SQL; even
with mutate, it is a very convoluted interface for attempting any
iteration more complex than the Babylonian Method. Implementation of a
GLM function with \pkg{sparklyr} iteration was initially planned, but the
point was already proven by something far simpler, and the point is one
that did not need to be laboured.

Ultimately, \pkg{sparklyr} is excellent at what it does, but convoluted and
inefficient when abused, as when attempting to implement iterated
functions.
