\subsubsection{Introduction}\label{introduction}

A problem exists in the architecture described in the prior section.
At base, the co-ordination between nodes in the
cluster requires significant effort from the master, with little room
for additional features, with all inter-node communication being remote
procedure calls only. This issue compromises nearly every aspect of
operations on distributed objects, particularly in the initial reading
in of data, and later data movement. The proposed amendments as
described in \cref{sec:amendments} provide some degree of amelioration, however the
root of the issue in requiring full co-ordinating facilities from the
master node is still not solved.

A potential solution exists in the use of message queues. Message queues
are commonly used for inter-process communication, consisting of queues
through which applications may communicate over \cite{curry2004message}.
The use of message queues for communication between nodes will allow for
significantly less knowledge required about other nodes within the
system, and enabling greater independence of action within each node.
Further benefits include allowing for asynchrony in more operations, the
ability to monitor the system externally through watching queues, as
well as the attendant benefits of decentralisation such as potentially
greater resilience and decreased central complexity.

Message queues are well established, seeing use from Operating Systems
to Web Services. For example, the QNX OS makes heavy use of message
queues for its microkernel architecture \cite{hildebrand1992qnx}. Tech
companies Stack Overflow and flickr also use message queues from redis
as central components of their infrastructure \cites{nolan2011flickr,montrose2016stack}. In this platform, the flexibility of Redis lists
and the availability of the \pkg{rediscc} package suggests the use of Redis in
the implementation of message queues \cite{sanfilippo2009redis,urbanek2020rediscc}. Alternatives include Apache \pkg{ActiveMQ}, \pkg{Kafka}, and
\pkg{Disque}, among others \cites{snyder2011activemq,garg2013kafka,sanfilippo2016disque}.

\subsubsection{Architecture Concept}\label{architecture-concept}

The concept retains the notion of data divided into uniquely identified
chunks, existing on nodes. The nodes each subscribe to queues dedicated
to chunks that they possess, undertaking action dependent on messages
received in their queue. In this way, nodes function as state machines,
reading messages, performing some operation (or not) depending on the
message content, and writing back in some form.

For example, some node has a chunk \(x\), and receives a message on the
\(x\) queue to add it to another chunk \(y\); if it didn't have the
chunk \(y\), it may post a message on the \(y\) queue, requesting it. It
is likely that the semantics are more general, and the initial message
of operation actually won't specify which chunk to add to \(x\), giving
it a more general request of addition between distributed objects, and
the node will have to determine for itself which chunks to pull to it,
if any.

A very simple example is given by \cref{lst:msg-q-master}
\cref{lst:msg-q-worker}, demonstrating the master and worker node routines
respectively.

\src{msg-q-master}{Master redis message queue}

\src{msg-q-worker}{Worker redis message queue}

The master simply pushes serialised calls to the appropriate queue, and
the worker loops reading messages from it's particular queue(s),
unserialising and evaluating any messages. The master node and the
worker node can be initialised in any order and with any time
difference, demonstrating the asynchrony. In this particular example,
the master puts out a call to add the number 1 to a predefined chunk
named ``chunk'', with the worker executing the call as expected. The
master doesn't have to have any knowledge about where the chunk exists,
and the worker likewise doesn't necessarily require information on where
the message originated from.

With relation to the queues, the aggregate functionality of nodes in a
cluster, can be considered distinctly to the functionality of a singular
node. The cluster must have some means of initialising the queues to
listen to for the individual nodes, for the reading in of an external
dataset. The process of operating based on queues is straightforward at
the outset, but requires considerable thought on the representation and
existence of objects other than the referent object of a queue when the
queues operations require those other objects.

The issue of data movement also requires consideration; while this is
largely an implementation-specific issue, it has a strong bearing on
conceptual architecture. These three considerations mirror the state
machine concept at an aggregate level, with resultant decisions
affecting the architecture at large.

\subsubsection{Plausible Extensions}\label{plausible-extensions}

While the use of message queues looks to ease many significant issues,
there are additional problems that it require addressing, primarily
resulting from the asynchrony and decentralisation.

Most pressingly, the issue of deadlock, in the context of data movement
in the process of nodes requesting and transferring data, a deadlock is
almost certain to occur if not dealt with. This specific instance may be
helped through careful implementation and more processing on the
initiator node (bringing it closer to a master node), but a possibly
superior extension is to have queue-adjacent servers on each node that
are able to operate concurrently with the main \R session that has as
it's sole purpose the transfer of chunks, leaving everything else to the
main \R session. These servers would also be ideal vertices at which to
implement data-duplication as a feature enabling redundancy in the
system

In a similar vein, failure detection can be implemented through a
concurrent ``heartbeat'' server, in the same manner as \pkg{Hadoop}
\cite{white2012hadoop}.

These extensions are worth bearing in mind for now, however the
considerations brought up in the prior section need to be answered
first.
