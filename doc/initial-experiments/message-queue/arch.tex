\subsubsection{Introduction}

The purpose of this section is to outline the workings of the message queue
architecture at the chunk layer.

\subsubsection{Overview}

The central units of this distributed system are a client, a queue server, and
a chunk server, as demonstrated in \cref{fig:chunk-arch}.

The client contains chunk references, which can be passed as arguments to
\code{do-call-chunkref}, alongside some function, in order to begin the
process of evaluation, and if assignment is desired, producing a new chunk
reference to continue computation with on the client end while the evaluation
continues on the other units.

\code{do-call-chunkref} composes a message based on the request, pushing that
to the queue identified by the chunk ID contained in the chunk reference, with
the queue existing on some central queue server.

The chunk server concurrently monitors all queues identified by the chunk ID's
of the chunks that the chunk server stores in a local chunk table.
It pops the message off the related queue, and has \code{do-call-chunk}
evaluate the function on the chunk, with various options determined by the
content of the received message.

The chunk server pushes some response message to the queue associated with that
particular job through a unique job ID, which may be picked up later by the
client later.

\fig{chunk-arch}{An outline of the architecture of the chunk component in the distributed system}

\subsubsection{Object Formats}
The fields belonging to a \code{chunkRef} object are the following:
\begin{description}
	\item[\texttt{CHUNK\_ID}] The name of the queue to post messages to, as
		well as the name of the chunk existing on the server to perform
		operations upon.
	\item[\texttt{JOB\_ID}] The name of the queue to pop a response from.
	\item[\texttt{RESOLUTION}] The status of whether a response has been
		heard from the server, taking the values ``UNRESOLVED'',
		``RESOLVED'', or a condition object signalling an error.
	\item[\texttt{PREVIEW}] A small preview of the complete object for use
		in printing.
\end{description}

Messages all belong to the \code{msg} class, and are currently categorised as
either requests, or responses, with the following fields:\\

Request:
\begin{description}
	\item[\texttt{OP}] Directive for server to carry out, e.g. ``ASSIGN''.
	\item[\texttt{FUN}] Function object or character naming function to
		perform on the chunk.
	\item[\texttt{CHUNK}] Chunk Reference for the server to attain
		information from.
	\item[\texttt{JOB\_ID}] The name of the queue to push a response to.
	\item[\texttt{DIST\_ARGS}] Additional distributed arguments to the
		function.
	\item[\texttt{STATIC\_ARGS}] Additional static arguments to the
		function.
\end{description}

Response:
\begin{description}
	\item[\texttt{RESOLUTION}] Resolution status; either ``RESOLVED'', or a
		condition object detailing failure due to error.
	\item[\texttt{PREVIEW}]  A small snapshot of the completed object for
		use in printing chunk references.
\end{description}

\subsubsection{Demonstration of Communication}

The \cref{tab:chunk-comm} shows a demonstration of verbose communication
between a client and a server, corresponding to the listings
\cref{src:chunk-client} and \cref{src:chunk-server} respectively.
In this demo, the server was started immediately prior to the client, being
backgrounded, and initial setup was performed in both as per the listings
referred to prior.

\longtab{chunk-comm}

\subsubsection{Next Steps}

The next step is to experiment with aggregates of chunks, as distributed objects.
A significant component of this involves point-to-point chunk movement, between multiple servers.
The package \pkg{osrv} looks to satisfy much of the infrastructure required for
this, with particular experiments to be dedicated specifically to establishing
a fast and reliable mechanism for co-ordination and data movement in the system.

