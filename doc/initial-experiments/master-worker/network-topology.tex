\subsubsection{Motivation}\label{sec:topo-motivation}

Central to the implementation of the definitive features of a
distributed system are the forms of structuring the network supporting it.
This is due to the fact that the primary constraint on such a system is
the distribution of data, along with the essential consideration of
message timing between nodes. While the algorithms in this particular
statistical system are equally necessary for functionality in the stated
aim of modelling and manipulation of large data, they are completely
dependent on the structure of the data such algorithms operate upon,
hence the primality placed on the system's information structures.

\subsubsection{Overview}\label{overview}

The first questions asked of a distributed system's information
structures relate to its topology and mechanism of message passing.
This particular system answers that necessarily through its context and
aims; that it is currently intended to exist as a largely transparent
platform within R, which necessitates interactivity and other standard R
development practices. Thus, a centralised structure must be chosen for
the system, with the node associated with the user's \R{} session acting as
the master. This is the form of operation of other distributed packages
currently existing in R, such as \pkg{sparklyr} and \pkg{SNOW}
\cites{luraschi20,tierney18}. This can be contrasted with a
decentralised system as in \pkg{pbdR} and its underlying
\pkg{MPI} implementation \cite{pbdR2012}, which requires \R{} programs
to be written in a manner agnostic to the nodes in which they are
executed upon, resulting in the same program being distributed to all
nodes in the system. In this way, the program is not capable of being
run interactively, something undesirable to the goals of our system. The
master will therefore always be local, all other nodes remote.

\subsubsection{Local}\label{local}

\subsubsection{Description of Current System}\label{description-of-current-system}

As it currently stands, the local information structures are entirely
described by S3 classes, the instances of which act as references to the
payload data being held remotely. These classes are composed as
environments, used for their mutable hash table properties, and contain
three elements:

\begin{description}
\tightlist
\item[\texttt{locs}]
a list of \pkg{RServe} connections through which the remote payload
data resides in discrete chunks
\item[\texttt{name}]
a UUID character scalar, which corresponds to the symbol which the
chunks are assigned to in the remote environment
\item[\texttt{size}]
an integer vector of the same length as \texttt{locs}, describing the
size (as in \texttt{nrow} or \texttt{length}) of the chunk at each
location
\end{description}

This is coupled with a garbage collection system consisting of a hook to
the removal of the reference object through \code{reg-finalize}. Upon
triggering the hook, a directive is issued to all chunks in
\texttt{locs} to remove \texttt{name}, thereby closing the loop between
creation and deletion on local and remote nodes.

\subsubsection{Motivation for Current System}\label{motivation-for-current-system}

The system exists in its current form primarily through motivations of
simplicity; minimising complexity in the system until further additions
are required. By themselves, \texttt{loc} and \texttt{name} are
sufficient for referencing any distributed chunks. \texttt{size} is
maintained for the regular need to know lengths of objects as part of
many standard operations, thereby reducing the lookup cost by keeping
the static information locally and directly attached to the reference.

The mechanism of garbage collection is likewise borne of simplicity and
necessity; it requires the least possible steps, and without it,
distributed chunks would accumulate remotely with no means of further
access if their reference is removed, essentially forming a high-level
memory leak.

\subsubsection{Insufficiency of Current Structures}\label{insufficiency-of-current-structures}

In spite of, and indeed because of, the simple information structure of
the local system, there remain aspects of the design that inhibit the
development of important features, many of them essential. In addition,
clarification in system semantics has revealed a need for greater focus
in areas presently under-served by the system.

A major feature lacking in this system is a global awareness of existing
connections, which can be used in preference to creating new connections
upon instantiation of a distributed object. Take for example, the act of
reading in successive distributed \file{csv}s into the system. The
first read takes in file location arguments, among others, then creates
new connections, finally returning a reference. The next read performs
exactly the same actions, and so on. This ignores the highly likely
situation where files are situated in the same locations, and
connections at those locations can be reused, thus potentially saving
from the overhead of extraneous connections and unnecessary data
movement of aligning objects with each other.

Another issue is the closing of connections; as it currently stands,
there is no appropriate garbage collection for connections.

The single name for all chunks also cuts out any possibility of having
multiple chunks belong to the same object referenced via a singular
connection, thereby cutting out a potential mechanism for arbitrary
indexing of objects.

\subsubsection{Proposal for New Structures}\label{sec:localproposal}

Significant enhancements to the system can be attained through
additional structures addressing the present deficits. Principally, the
introduction of a central table of connections will serve as a single
source of truth, avoiding issues of non-knowledge in creation, deletion,
and usage of channels. This would require a change in the structure of
reference objects, and can consist in changing literal \pkg{RSclient}
channel pointers to identifiers to be searched for in the central
location table. In this way it provides a solution in the manner of the
fundamental theorem of software engineering:

\cqu{oram2007beautiful}.

The table slots corresponding to each identifier may also contain
relevant information to the connection such as host name, rack, etc., in
order to optimise data movement, as well as aid in the decision of
whether or not to create new connections for newly read or instantiated
objects.

Additional improvements, though unrelated, include changes to the
reference classes to allow for globally unique names for each chunk,
which will allow the same connection to house multiple chunks of a
cohesive distributed object, thereby enabling arbitrary indexing
operations. With such changes in structure, garbage collection is able
to be enhanced through centralising the objects of garbage collection
within the central table of locations.

One potential algorithm for garbage collection could involve marking
table elements with chunks to be removed, at their associated channel,
as part of a reference garbage collection hook. The marked objects can
then be used as part of a directive for remote removal at the next
convenience. This can be combined with a reference counter of the number
of extant objects at the referent environment of each channel; upon
complete emptying of the environment, signified by a counter of zero,
that channel itself may then be closed and removed.

\subsubsection{Relation to Existing Systems}\label{sec:localrel}

Most other distributed systems in \R{} require manual specification of a
cluster that then operates either in the background or as an object that
must be retained and manipulated. What is described here bears closer
resemblance to a file system than any particular distributed \R{} package,
with particular relation to the UNIX file system
\cites{ritchie1979evolution,thompson1974unix}. In the UNIX
file system, files contain no additional information beyond what is
written to them by the user or file generation program. Directories are
also files and consist solely of a regular textual mapping from a file
name to its entry (\emph{inode}) in a central system table
(\emph{ilist}). The inode contains metadata associated with a file such
as access times and permissions, as well as the physical address of the
file on disk.

To analogise, references in our system are equivalent to directories.
They provide a mapping from connection names (files) to their entries
(inodes) in a central table (ilist). Furthermore, the table entry
contains an \pkg{RSclient} pointer, analogous to a disk address, as
well as metadata. The form of the metadata differs due to separate
priorities; a list of chunk names in the place of permission bits, etc.
In theory this also allows copies of references to behave as hard links,
though this will introduce major issues involving synchronisation, and
is therefore be avoided for now. This form of garbage collection bears
some resemblance to the file system garbage collection as well, in that
inodes count the number of links to them, issuing a removal directive at
zero links, though our system supplements this through collecting
specific names of chunks for second degree removal. In this manner, the
``marking'' via name collection is closer to the method of marked
garbage collection, in conjunction with reference counting
\cite{knuth1}.

\subsubsection{Remote: Description and Motivation for Current System}\label{description-and-motivation-for-current-system}

The remote end of the system is the simplest component of the entire
setup. Each remote \R{} process is hosted through
\pkg{RServe}, and accessed through \pkg{RSclient} on the local
end. The remote \R{} process holds chunks of data in its global
environment, and performs whichever operations on that data as are
directed to it from the master \R{} session. The data possesses no more
structure than what was already in the chunk following reading,
operation upon, or reception by the node. This has again been due to
reasons of simplicity, as no presuppositions of structure suggested
themselves at the outset.

\subsubsection{Insufficiency of Current Structures}\label{insufficiency-of-current-structures-1}

The system works very well for something general purpose. However, it
ignores much of the structure inherent in common primitive \R{} objects
such as vectors. For example, to numerically index elements of a
distributed vector, an indexing algorithm currently translates the index
numbers into node-specific indices, and forwards those translated
indices on as part of a call to the relevant nodes. This sees issue when
there are disparate elements at a particular node selected between the
elements of other nodes, and the mechanism for numerical translation
breaks down.

\subsubsection{Proposal for New Structures}\label{proposal-for-new-structures}

A potential mechanism for improvement is to attach index attributes
corresponding to the overall index to the chunks. In combination with
remotely-run routines, the local session simply needs to send out a
request for particular indices to all of its connections, and they can
work out themselves which elements, if any, they correspond to,
returning a vector of elements matched, for us in the creation of a new
reference locally.

This would certainly solve the problem, however it may be redundant to
simply allowing local index translation to account for multiple chunks
at a single connection (as described in \cref{sec:localproposal}). It certainly uses significantly more
messaging bandwidth, though by distributing processing of index
translation across nodes, it may be faster in practice. In addition, the
additional structure forced on data chunks by attaching indices is
somewhat contrary to the lack thereof in the analogous UNIX
filesystem described in \cref{sec:localrel}.

\subsubsection{Further Research}\label{further-research}

Further work involves the actual implementation and assessment of the
proposed information structures, as part of a general rewrite.
Additional research may also involve other garbage collection systems,
with especial interest in file systems, such as those of the Inferno and
Plan 9, distributed operating systems borrowing heavily from
UNIX \cites{dorward1997inferno,pike1995plan}.
