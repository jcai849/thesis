\subsubsection{Motivation}\label{motivation}

To create a minimal implementation of distributed objects in R, with
transparent operations defined, in order to ascertain relevant
associated issues with further work on distributed computations using R.

\subsubsection{Method}\label{method}

Using the nectar cluster, the \code[sh]{hdp} node was used as a master
with which to control worker nodes \code[sh]{hadoop1} through to
\code[sh]{hadoop8}. \pkg{RServe} was used as the means for control and
communication with the workers. S3 classes were defined for
\code{cluster}, \code{node}, \code{distributed-object} and
\code{distributed-vector}. Communication functions operate serially,
but were written with future parallelisation and speed in mind.

The \code{node} class contains information on connections to the
worker nodes. The \code{cluster} class is a collection of
\code{node}s. The cluster is set up using \code{make-cluster},
which \code[sh]{ssh} into the hosts and launches RServe, along with the
relevant libraries and functions. The global environment local to
specific nodes can be checked with the \code{peek} method, serving
purely as a sanity check at present. \code{send} is a generic with
methods defined for the \code{node} and \code{cluster} classes; it
takes objects from the master node and partitions the objects into
equally sized consecutive pieces and distributes them to the hosts
referenced by the \code{cluster} or \code{node} objects. It can
equally handle objects with smaller \code{split} sizes than there are
nodes, dispersing them maximally. \code{send} is used just to get
data to the nodes to bootstrap the system, and wouldn't be used by the
end-user.

\code{distributed-object} at present has no methods defined, serving
as a placeholder for an abstract distributed class.
\code{distributed-vector} inherits from \code{distributed-object},
and serves as a master reference to data that may be spread across
multiple nodes. It contains a list of hostnames, the indices of the
vector residing on each node, and the name of the vector on the nodes,
typically being a UUID generated with the \code{distributed-vector}
creation.

\code{receive} is the complement to \code{send}, giving a
\code{distributed-object} as an argument, and receiving the unsplit
referent of the \code{distributed-object} as the value. The method
will have additional usage as a remote version, which would enable
point-to-point communication through a node calling \code{receive} on
some distributed object, thereby requesting the referent from its
location on all other nodes. Such remote usage is not yet implemented
due to difficulties with point-to-point communication using RServe.
However, such functionality is essential, and is discussed further in
the successive sections.

As a means of testing operations between \code{distributed-vector}
objects, S3 \code{ops} methods were defined, using a complex quoting
function in order to call the correct \code{generic} and reference
the name of the vectors on the worker nodes. They can interact with
non-distributed objects, with the non-distributed objects being coerced
to distribted. To enable interaction between vectors of different
lengths, some means of alignment must be defined, to allow elements at
equivalent positions to be processed at the same node. This is still to
be implemented, with further discussion given in the next section.

No quality-of-life methods such as \code{print} were defined, with
error-checking and special case consideration being kept to a minimum,
due to the primarily exploratory nature of the implementation.

\subsubsection{Relevant Points of Interest}\label{relevant-points-of-interest}

Already, the experiment has raised several very important considerations
that had not been noted prior.

Memory management was a particular concern; management of reference and
location of distributed objects emulates memory management at a much
lower level, introducing similar issues to those encountered in
systems-level programming.

The initial distribution of objects raises questions of appropriate
algorithms that take load-balancing and other factors into
consideration. One particular example is the question of what to do with
vectors of different length in their distribution across nodes; if split
equally across nodes, it is unlikely that elements at corresponding
posiions between the vectors, and for operations to take place, a
significant amount of data movement (``shuffling'') will have to take
place. Consideration should be given to forms of distribution that
minimise data movement, perhaps through maximisation of correspondence
with existing vectors, while still avoiding misbalancing node memory.

Memory leaks, not much of a problem at the \R level with garbage
collection, return to a potential problem with assignment of distributed
objects being fixed to their local \R processes. For example, with the
following code consisting of distributed vectors:
\code{abc}, what occurs is that on every node
\code{a} and \code{b} exist on, they are summed together, with the
result saved as a new variable with a UUID name; a reference to the name
and locations is then stored locally in the variable \code{c}. Were
\code{a} and \code{b} not to be assigned, however, the result would
still be saved on all of the worker nodes, taking up memory, but without
any local handle for it.

This is a memory leak at a high level, and reassignment is even worse;
conceivably, there could be some side effect for the cases of
non-assignment and reassignment, though this would require a level of
reflection whose existence is currently unclear in R.

Dealing with objects of greater complexity such as matrices are certain
to pose problems, and it is unlikely that whatever evolution of this
implementation would perform better than something that has had years
and teams worth of effort poured into it, such as LAPACK or SCaLAPACK.

The need for data movement between nodes as in the case of aligning two
vectors to exist at equivalent positions at equivalent nodes for the
sake of processing, if it is to be done efficiently, requires
point-to-point communication. The alternative is to have each node
channel data through the master and then on to the appropriate node,
which would be a massive waste of resources. This point-to-point
communication is not so easy to perform in reality, as RServe forks a
fresh \R session at every new connection, so objects that exist in a
particular node in connection with the master are not able to be
referenced in any other connection.

\subsubsection{Next Steps}\label{next-steps}

The next steps in this experiment should involve introducing quality of
life aspects to distributed objects such as formal getters and setters,
before it becomes unmanageable. Further methods for
\code{distributed-vector} as well as a generalisation to vectors of
different lengths are necessary. The implementation of operations
between vectors of different lengths requires elements of vectors at
equivalent positions to be on the same node for processing; this implies
some kind of \code{align} method, which as discussed in the previous
section, would ideally require point-to-point communication, which isn't
so easily permitted through (ab)using RServe. In turn, some custom
solution would likely be required. Upon implementing this, the system
will be highly flexible, with a clean demonstration of this begging for
the right methods defined such that \code{summary} and the like work
smoothly. This would lead naturally to the definition of
\code{distributed-data-frame} objects and the like. Furthermore, a
means of reading data from distributed storage to their local R
processes would likely yield very worthwhile insights to the process of
creation of a distributed \R system. Porting to S4 may be worthwhile, but
it can be performed later.
