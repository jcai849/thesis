\hypertarget{introduction}{%
    \subsection{Introduction}\label{introduction}}

Statistics is concerned with the analysis of datasets, which are
continually growing bigger, and at a faster rate; the global datasphere
is expected to grow from 33 zettabytes in 2018 to 175 zettabytes by
2025\cite{rydning2018digitization}.

The scale of this growth is staggering, and continues to outpace
attempts to engage meaningfully with such large datasets. By one
measure, information storage capacity has grown at a compound annual
rate of 23\% per capita over recent decades\cite{hilbert2011world}. In
spite of such massive growths in storage capacity, they are far
outstripped by computational capacity over time \cite{fontana2018moore}.
Specifically, the number of components comprising an integrated circuit
for computer processing have been exponentially increasing, with an
additional exponential decrease in their cost\cite{moore1975progress}.
This observation, known as Moore's Law, has been the root cause for much
of computational advancement over the past half-century. The
corresponding law for computer storage posits increase in bit density of
storage media along with corresponding decreases in price, which has
been found to track lower than expected by Moore's law metrics. Such
differentials, between the generation of data, computational capacity
for data processing, and constraints on data storage, have forced new
techniques in computing for the analysis of large-scale data.

The architecture of a computer further constrains the required approach
for analysis of big data. Most general-purpose PC's are modelled by a
random-access stored-program machine, wherein a program and data are
stored in registers, and data must move in and out of registers to a
processing element, most commonly a Central Processing Unit (CPU). The
movement takes at least one cycle of a computer's clock, thereby leading
to larger processing time for larger data.

Reality dictates many different forms of data storage, with a Memory
Hierarchy ranking different forms of computer storage based on their
response times\cite{toy1986computer}. The volatility of memory (whether
or not it persists with no power) and the expense of faster storage
forms dictates the design of commodity computers. An example of a
standard build is given by the Dell Optiplex 5080, with 16Gb of Random
Access Memory (RAM) for fast main memory, to be used as a program data
store; and a 256Gb Solid State Drive (SSD) for slow long-term disk
storage\cite{cornell2021standardcomp}. For reasonable speed when
accessing data, a program would prioritise main memory over disk storage
- something not always possible when dataset size exceeds memory
capacity, larger-than-memory datasets being a central issue in big data.
A program that is primarily slowed by data movement is described as
I/O-bound, or memory-bound. Much of the issue in modelling large data is
the I/O-bound nature of much statistical computation.

The complement to I/O-bound computation is computation-bound,
wherein the speed (or lack thereof) is determined primarily through the
performance of the processing unit. This is less significant in
large-scale applications than memory-bound, but remains an important
design consideration when the number of computations scale with the
dataset size in any nontrivial algorithm with greater than
\(\mathcal{O}(1)\) complexity.

The solution to both memory- and computation-bound problems has largely
been that of using more hardware; more memory, and more CPU cores. Even
with this in place, more complex software is required to manage the more
complex systems. As an example, with additional CPU cores, constructs
such as multithreading are used to perform processing across multiple
CPU cores simultaneously (in parallel).

The means for writing software for large-scale data is typically through
the use of a structured, high-level programming language. Of the myriad
programming languages, the most widespread language used for statistics
is R. As of 2021, R increased in popularity to rank 9th in the TIOBE
index. R also has a special relevance for this proposal, having been
initially developed at the University of Auckland by Ross Ihaka and
Robert Gentleman in 1991\cite{ihaka1996r}.

Major developments in contemporary statistical computing are typically
published alongside R code implementation, usually in the form of an R
package, which is a mechanism for extending R and sharing functions. As
of March 2021, the Comprehensive R Archive Network (CRAN) hosts over
17000 available packages\cite{team20:_r}. Several of these packages are
oriented towards managing large datasets, and will be assessed in
\ref{sec:local}\ref{sec:dist} below. This project aims to develop an R
package that provides a means for writing software to analyse very large
data on clusters consisting of multiple general-purpose computers.