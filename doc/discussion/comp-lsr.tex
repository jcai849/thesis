With regards to other packages, it may be worth returning to the early \cref{tab:feature-comp}, in comparing feature by feature.
The following description details features from the table and explanations of \lsr{}'s relation to them.

\begin{description}
\item[Distributed Computation] \lsr{} holds the capability for distributed computation as a core component of the provided API. This is given by the \code{do-ccall} and \code{do-dcall} functions at the chunk and distributed object levels respectively.
\item[Evaluation of User-Specified Code] User-specified code is similarly run via the given \code{do-ccall} and \code{do-dcall} functions.
\item[Native Support for Iteration] The combination of the user-specified code functions with the built-in garbage collection allows for arbitrary looping without memory issues.
\item[Object Persistence at Nodes] The chunk and distributed object concepts in \lsr{} depend on exactly this property.
\item[Support for Distributed File Systems] Local filesystems over separate hosts are supported. Reading from HDFS or the like is not currently a core component of the API, but could easily be extended to allow for it.
\item[Ease of Setup] Initialisation functions are included with the framework; all that is required is the existence of the framework as installed on all hosts, as well as a reliable ssh connection to them.
\item[Inter-Node Communication] The movement from the message-queue-based approach to the final direct approach facilitated this.
\item[Interactive Usage] Interactivity has been a capability from the beginning.
\item[Backend Decoupling] As a new framework, there is only one backend available, which is provided by \orcv{}. The layered approach allows for any other backend that provides the \orcv{} interface to be swapped in.
\item[Evaluation of Arbitrary Classes] This is a core feature that drives the means of combination of emerged distributed objects.
\item[Package-specific API] Unique API, following the principle of least surprise, with naming and semantics similar to base \R{}.
\item[Methods for Standard Generics] Some provided; e.g. \code{summary}, \code{ops}, etc.
\item[Methods for dplyr Generics] Provided by the \lso{} package, including true \code{group-by}.
\end{description}

It is clear to see from this that \lsr{} fills out the selected features like no other package.
The closest is the \pkg{pbdR} package framework, and it is worth describing the core differences at least briefly.
Chief among them, the main difference between \pkg{pbdR} and \lsr{} is the backend, and how the package is defined by it.
\pkg{pbdR} -- specifically, \pkg{pbdDMAT} -- is intimately tied to \pkg{MPI}. 
While \pkg{pbdZMQ} specifies an interface to \pkg{ZeroMQ}, it defines a rather low-level interface and isn't able to swap in place of \pkg{pbdMPI} to support the higher-level components of \pkg{pbdDMAT}.
Because of this coupling, the abstraction leaks through and even the higher-level packages such as \pkg{pbdDMAT} end up reflecting interaction patterns typical of \pkg{MPI}.
For example, all programs are taken to be written in SPMD style.
The coupling with \pkg{MPI} comes with benefits and drawbacks relative to no dependence on any external backend, as in \lsr{}.
The range of the \pkg{MPI} ecosystem may be used to support development efforts if well-integrated.
The need to set up \pkg{MPI} is not too onerous, but from there it becomes increasingly important to know the inner workings of \pkg{MPI} when edge-cases or bugs are encountered, which is a standard part of any non-trivial program development.
In addition, while it is plausible that \pkg{pbdDMAT} could be written to be interactive, it currently isn't, which is a ringing difference that can't be overlooked.
It may be that the \pkg{MPI} ecosystem proves too valuable to give up in more complex cases.
If so, then \lsr{} can at least be said to get very far with most cases, with the simple interface and minimal mental overhead allowing rapid prototyping.
When optimisation down to the chunk grid level matched to the machine is required, then armed with a clearer understanding of the problem due to the use of \lsr{} for prototyping, then \pkg{pbdR} may be tried.
The difference in complexity stands as a point in its own right -- the source code supporting \pkg{pbdR} is orders of magnitude larger than that of \lsr{} -- again, this has practical consequences, including robustness and mental overhead, that have to be evaluated by the end user.
In spite of the differences, it is of great interest where similarities lie.
Much of this effort may even be seen as some kind of validation of many of the \pkg{pbdR} principles, wherein starting from scratch and reasoning over a constructive interface description, many shared interface components are arrived at independently.

