Recall the principal problem driving this project: big data.
Big data -- and specifically, data that is larger than commodity hardware memory -- possesses significant value in its analysis.
Any statistical analysis benefits from a well-written statistical algorithm, especially the most innovative analyses making use of totally novel algorithms.
Such novel algorithms are typically complex and iterative, and require a comfortable means of expression and extension.
\Cref{ch:lit-review} explored many of the existing options, and while many are attractive, none were entirely sufficient to bridge this gap between big data and expressive statistical algorithms in \R{}.
The \lsr{} project attempted to bridge this gap, and in many ways it found success.
As with any large, complex project, there remain many things that could have, and should have, been done differently.


