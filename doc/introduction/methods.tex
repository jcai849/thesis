The structure of the \lsr{} system has been defined principally by the response to linguistic challenges facing an API for modelling on big data.

The challenge of the object system provokes questions of what objects should comprise such a system, and what properties they should possess.
\lsr{} has answered this in a standard fashion, providing chunks, references to the chunks, and an abstraction over them for the end-user.
Users interact indirectly with chunks, by way of chunk references, which are typically collected as arrays and made opaque to the high-level user, as shown in \cref{fig:distobjref}.
Arbitrary underlying data, a layered class hierarchy for data access, and asynchronous and immediate distributed procedure dispatch are further core design decisions.
For example, assume that some dataset is split into four pieces row-wise, with the first two pieces residing on two separate processes, and the last two on a single process.
Such a topology is capable of being represented in the \lsr{} system, with diagrammatic representation as in \cref{fig:top}.

\fig{distobjref}{Distributed object, showing chunks and their references across disparate nodes.}

\fig{top}{Example distributed object reference relations}

Communication structure is a further challenge, which has as its answer an implicit virtual topology of the distributed system.
\lsr{} engages distributed worker nodes in a peer-to-peer fashion, chunks being the core means of addressing, with their location made opaque at the user-level of the system.

Concurrency is an essential component to distributed systems, and this challenge saw its response as both in-process and inter-process concurrency with \lsr{}; the base supporting layer of the \lsr{} stack is a bespoke in-memory TCP message queue service that allows for communication between nodes concurrent with their other operations.
Between nodes, routines run asynchronously over chunks, with parallelism implicit and controlled in distributed fashion.

Evaluation and scope may serve to speciate languages, as in the case of \R{} from \proglang{S}, and they take special forms in a distributed system.
\lsr{} seeks to minimise any differences from the \R{} language, so as to provide as transparent an experience for developers as possible, but with respect to evaluation, differs in following a call-by-value, as opposed to call-by-need pattern.
Furthermore, errors can be caught, but are only propagated to the caller upon emergence of the underlying chunks, with the system favouring asynchrony to strictness.
Scope is likewise limited in favour of message transfer efficiency.

In order to enable complex and iterative models, a distributed garbage collection system is also essential.
Such a system should handle mutable underlying chunk data as well.
Mutable data is treated equivalently to immutable data, where all operations on chunks result in new references, with new identifiers, that are surjective to their referent chunks.
\lsr{} enables garbage collection in an efficient and conservative manner, through automatically keeping track of the directed acyclic graph of chunk generation history alongside each chunk reference, and clearing this upon proof of computation.
Chunks lacking references are marked for deletion and may then be removed by \R{}'s internal garbage collection.

