Statistics is concerned with the analysis of datasets, which are
continually growing bigger, and at a faster rate; the global datasphere
is expected to grow from 33 zettabytes in 2018 to 175 zettabytes by
2025\cite{rydning2018digitization}.

The scale of this growth is staggering, and continues to outpace
attempts to engage meaningfully with such large datasets. By one
measure, information storage capacity has grown at a compound annual
rate of 23\% per capita over recent decades\cite{hilbert2011world}. In
spite of such massive growths in storage capacity, they are far
outstripped by computational capacity over time \cite{fontana2018moore}.
Specifically, the number of components comprising an integrated circuit
for computer processing have been exponentially increasing, with an
additional exponential decrease in their cost\cite{moore1975progress}.
This observation, known as Moore's Law, has been the root cause for much
of computational advancement over the past half-century. The
corresponding law for computer storage posits increase in bit density of
storage media along with corresponding decreases in price, which has
been found to track lower than expected by Moore's law metrics. Such
differentials, between the generation of data, computational capacity
for data processing, and constraints on data storage, have forced new
techniques in computing for the analysis of large-scale data.

The architecture of a computer further constrains the required approach
for analysis of big data. Most general-purpose PC's are modelled by a
random-access stored-program machine, wherein a program and data are
stored in registers, and data must move in and out of registers to a
processing element, most commonly a Central Processing Unit (CPU). The
movement takes at least one cycle of a computer's clock, thereby leading
to larger processing time for larger data.

Reality dictates many different forms of data storage, with a Memory
Hierarchy ranking different forms of computer storage based on their
response times\cite{toy1986computer}. The volatility of memory (whether
or not it persists with no power) and the expense of faster storage
forms dictates the design of commodity computers. An example of a
standard build is given by the Dell Optiplex 5080, with 16Gb of Random
Access Memory (RAM) for fast main memory, to be used as a program data
store; and a 256Gb Solid State Drive (SSD) for slow long-term disk
storage\cite{cornell2021standardcomp}. For reasonable speed when
accessing data, a program would prioritise main memory over disk storage
- something not always possible when dataset size exceeds memory
capacity, larger-than-memory datasets being a central issue in big data.
A program that is primarily slowed by data movement is described as
I/O-bound, or memory-bound. Much of the issue in modelling large data is
the I/O-bound nature of much statistical computation.

The complement to I/O-bound computation is computation-bound,
wherein the speed (or lack thereof) is determined primarily through the
performance of the processing unit. This is less significant in
large-scale applications than memory-bound, but remains an important
design consideration when the number of computations scale with the
dataset size in any nontrivial algorithm with greater than
\bigO{1} complexity.

The solution to both memory- and computation-bound problems has largely
been that of using more hardware; more memory, and more CPU cores. Even
with this in place, more complex software is required to manage the more
complex systems. As an example, with additional CPU cores, constructs
such as multithreading are used to perform processing across multiple
CPU cores simultaneously (in parallel).

The means for writing software for large-scale data is typically through
the use of a structured, high-level programming language. Of the myriad
programming languages, the most widespread language used for statistics
is R. As of 2023, \R{} ranks {18} in the TIOBE index.
\R{} also has a special relevance for this thesis, having been
initially developed at the University of Auckland by Ross Ihaka and
Robert Gentleman in 1991\cite{ihaka1996r}.

The rate of growth of datasets continues to outpace attempts to engage meaningfully with them, as individual computer memory limits are increasingly exceeded \cite{kleppmann2017dataintensive}.
At the scale of big data, speed also becomes a constraining factor, with concurrency and parallelism being of increasing importance.
The aim of a statistician seeking to gain novel insight from such datasets commonly includes the interactive use of a complex statistical model, often implemented from scratch using \R{}.
No single system satisfactorily provides the capacity to meet this demand.

Those systems that do come close to meeting the demand provide direction regarding how to gain insight from larger-than-memory datasets.
Most importantly, the standard solution for handling big data is to operate over a distributed system~\cite{boja2012distributed}.
Several systems have seen widespread use within the context of data and machine learning pipelines, such as \pkg{Spark}~\cite{zaharia2016apache} and \pkg{Hadoop}~\cite{shvachko2010hadoop}.
For the statistician mostly familiar with \R{}, these systems provide APIs to \R{} where distributed data may be manipulated and pre-made models fitted.
However, these API's are often found lacking when attempted to be used for the creation of complex statistical models that don't come pre-packaged, due to this not being their primary use-case, and \R{} not being their target language.

Ultimately, the problem is that of practically expressing a statistical algorithm for larger-than-memory data in \R{}.

This thesis outlines and details the attempts a solution through the \lsr{} framework, which is intended as a proof-of-concept solution to the problem of expressing distributed statistical algorithms in \R{}.

Within the motivating context provided, the \lsr{} project has sought to provide a full stack for working with larger-than-memory data in \R{}, allowing the developer to manipulate distributed data and create arbitrary complex, iterative models with which to fit to the data, over a self-contained user-specified computing cluster.

\section{Methods}\label{sec:methods}
The structure of the \lsr{} system has been defined principally by the response to linguistic challenges facing an API for modelling on big data.

The challenge of the object system provokes questions of what objects should comprise such a system, and what properties they should possess.
\lsr{} has answered this in a standard fashion, providing chunks, references to the chunks, and an abstraction over them for the end-user. Users interact indirectly with chunks, by way of chunk references, which are typically collected as arrays and made opaque to the high-level user, as shown in \cref{fig:distobjref}. Arbitrary underlying data, a layered class hierarchy for data access, and asynchronous and immediate distributed procedure dispatch are further core design decisions.
For example, assume that some dataset is split into four pieces row-wise, with the first two pieces residing on two separate processes, and the last two on a single process.
Such a topology is capable of being represented in the \lsr{} system, with diagrammatic representation as in \cref{fig:top}.

\fig{distobjref}{Distributed object, showing chunks and their references across disparate nodes.}

\fig{top}{Example distributed object reference relations}

Communication structure is a further challenge, which has as its answer an implicit virtual topology of the distributed system.
\lsr{} engages distributed worker nodes in a peer-to-peer fashion, chunks being the core means of addressing, with their location made opaque at the user-level of the system.

Concurrency is an essential component to distributed systems, and this challenge saw its response as both in-process and inter-process concurrency with \lsr{}; the base supporting layer of the \lsr{} stack is a bespoke in-memory TCP message queue service that allows for communication between nodes concurrent with their other operations.
Between nodes, routines run asynchronously over chunks, with parallelism implicit and controlled in distributed fashion.

Evaluation and scope may serve to speciate languages, as in the case of \R{} from \proglang{S}, and they take special forms in a distributed system.
\lsr{} seeks to minimise any differences from the \R{} language, so as to provide as transparent an experience for developers as possible, but with respect to evaluation, differs in following a call-by-value, as opposed to call-by-need pattern. Furthermore, errors can be caught, but are only propogated to the caller upon emergence of the underlying chunks, with the system favouring asynchrony to strictness. Scope is likewise limited in favour of message transfer efficiency.

In order to enable complex and iterative models, a distributed garbage collection system is also essential.
Such a system should handle mutable underlying chunk data as well.
Mutable data is treated equivalently to immutable data, where all operations on chunks result in new references, with new identifiers, that are surjective to their referent chunks.
\lsr{} enables garbage collection in an efficient and conservative manner, through automatically keeping track of the directed acyclic graph of chunk generation history alongside each chunk reference, and clearing this upon proof of computation.
Chunks lacking references are marked for deletion and may then be removed by \R{}'s internal garbage collection.


\section{Results}\label{sec:results}
\lsr serves as a functioning system, capable of performing complex statistical analyses over larger-than-memory datasets.
The implementation of this system makes use of a layered approach, wherein each layer targets a different category of user.
A description of the implementation structure of \lsr follows.

The system is supported at the base layer by the \pkg{orcv} package, which exists as an in-memory threaded TCP message queue.
It was created specifically for \pkg{largescaler}, making use of the \proglang{C} API for \R, though it is sufficiently general to serve the wider purpose of a message queue for the transfer of \R objects between \R processes.
Central to the functionality of \pkg{orcv} is its multithreaded operation, allowing transfers to take place in the background of the host \R process, thereby not blocking computation.
The core userbase of this package is intended as developers on \lsr.

Sitting on top of \pkg{orcv}, the package \pkg{chunknet} enables the creation of detached nodes, which use \pkg{orcv} to communicate, and operate using their own event loops populated via the queue provided by \pkg{orcv}.
The instances of these nodes provided by the package are worker nodes, and a location service, which serve to operate on and locate chunks, respectively.
The client interface is also provided by \pkg{chunknet}, allowing interaction with chunks as the major user-facing class in this package.
Chunks can be interacted with individually, or collected as part of arbitrary-dimension arrays, over which distributed \code{apply()}'s and the like are defined.
The main users of this package are intended to be power-users of distributed statistical algorithms who seek to maximise performance.

The package that meets the demanding statistician as referenced is given by the \lsr package, which offers the distributed object as an abstracted class where chunk distribution is handled implicitly by the package, freeing the statistician to focus on model creation. Further features offered by the package include distributed environment setup, an automatic distributed function converter, distributed functional programming tools such as a reduce operator, distributed i/o, checkpointing, shuffling of datasets with implicit load-balancing, and a \pkg{dplyr} interface to distributed objects.

\section{Discussion}\label{sec:discussion}
The \lsr{} system has been proven over a number of application areas.
Data manipulation is a basic necessity, as it is required for modelling, and is provided well by other systems.
\lsr{} is capable of a full set of data manipulations, including all that are provided by the \pkg{dplyr} package.
Model fitting is demonstrated in the proof-of-concept \pkg{largescalemodels} package, which includes a variety of models, including linear models and generalised linear models.
Work is currently underway to develop examples of boosted models, as well as a convex optimisation methods such as the alternating direction method of multipliers.

The scope for future work remains significant, enabled by the high level of extensibility provided by the system.
External systems which serve to monitor performance or take up the role of garbage collection would grant the possibility of greater reliability.
Robustness could be gained through self-healing datasets, a potential that has a precedent in a current prototype, which allows for resilience to node failure in a more efficient manner than that of the current Resilient Distributed Datasets~\cite{zaharia2012resilient}.
Further resilience can be gained within the system through operating the location service as a distributed hash table, leaving no central point of failure in a fully peer-to-peer system.

