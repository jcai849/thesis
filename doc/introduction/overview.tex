This thesis can be thought of as in three parts, based on what is the most important pieces to be gained from this project.

First, to gain context.
This chapter, in its role for this part, has laid down the problem: expressing statistical algorithms at scale in \R{}.
From here, the literature review as \cref{ch:lit-review} serves to describe the broad field of large-scale statistical computing, as it relates to the \lsr{} project.
The literature review extends from a broad review to include several case studies of some of the more eminent projects by which to understand the problem.
With an understanding of the external context, \cref{ch:constructive-description} attempts to formulate a constructive description of the attributes any solution should have, in solving \cref{ch:introduction}'s problem.

The next part is to directly experiment.
To discern first-hand, through replication of facets of existing literature, and attempts at combinations and new forms, what, empirically, lends itself well to the solution.
\Cref{ch:initial-experiments} supports this part, describing two separate systems that were created as experimental vessels by which to clarify new forms of large-scale statistical modelling systems in \R{}.

Finally, the result; the final iteration of experimentation, cohered into a single \lsr{} framework.
\Cref{ch:ui,ch:lasso,ch:implementation} describe the user interface, a use-case, and implementation details of this system, respectively.
The user interface is the core offering and novelty, with emphasis being placed on expressiveness and extensibility.
Likewise, the use-case is intended to illustrate this interface, with implementation details being secondary, while still offering motivating details for certain design decisions.

Following these parts, a discussion on the delivered platform is given in the concluding \cref{ch:discussion}, which serves to relate the \lsr{} system back to the originating context and problem, with a critical assessment and recommendations for future research.
