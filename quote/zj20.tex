\pkg{disk.frame} works by breaking large datasets into smaller individual
chunks and storing the chunks in fst files inside a folder. Each chunk
is a fst file containing a data.frame/data.table. One can construct the
original large dataset by loading all the chunks into RAM and row-bind
all the chunks into one large data.frame. Of course, in practice this
isn't always possible; hence why we store them as smaller individual
chunks. \pkg{disk.frame} makes it easy to manipulate the underlying chunks
by implementing dplyr functions/verbs and other convenient functions
(e.g.~the \ccode{cmap} function which applies the function fn to each chunk of a.disk.frame in
parallel). So that \pkg{disk.frame} can be manipulated in a similar
fashion to in-memory data.frames.
